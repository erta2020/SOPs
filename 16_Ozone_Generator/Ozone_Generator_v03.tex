\documentclass[12pt]{../SOP2}
\usepackage[english]{babel}
\usepackage{blindtext}
\usepackage{lipsum}

%\documentclass{article}

%\documentclass[12pt]{~/github/SOPs/SOP_Template/SOP}

\title{Ozone Generator}
\date{02/02/2017}
\author{Marc Los Huertos}
\approved{Marc Los Huertos}
\ReviseDate{\today}
\SOPno{20}

\usepackage{Sweave}
\begin{document}
\Sconcordance{concordance:Ozone_Generator_v03.tex:Ozone_Generator_v03.Rnw:%
1 16 1 1 0 67 1}


\maketitle

\section{Scope and Application}

\NP The scope of this SOP is train researchers in how to effecively use the Laboratory Ozonoe Generator 

\NP As a researcher, occassions will arrise in which fabricated ozone gas must be used in order to test a hypothesis. Through using this generator, researchers can create ozone gas in small quantities, from dry air or oxygen and with negative or positive pressures. 

\section{Summary of Method}

\section{Definitions \& Parts}

\NP Electronis Power Board: This unit utilises a high frequency/high voltage power board which is designed for continuous operation and the circuitry has been specifically designed to incorporate "fold-back" protection which senses if a higher than normal current is being drawn by the ozone generating module and automatically reduces the power to protect itslef from overload or short circuiting. It alos incorporates variable frequency control which is utilised to vary the ozone production between 5-100 percent which is controlled via a manual control knob located on the front panel of the unit. 

\NP Ozone Generating Module: The air cooled ozone generating module is located at the rear of the enclosure and is incorporated within a 316 stainless steel shrouded housing which surrounds the cooling fan. The shrouded housing has thermostatic protection to stop ozone production if the module temperatures rises due to failure of the cooling fan.
\NP The module consits of a ceramic dielectirc with 316 stainless steel electrode continaed within an aluminium finned outer heatsink housing. The module end caps are manufactured from PTFE and PVDF tubing connectors. 
  
\NP Ozone Generator Enclosure: The enclosure incorporates ventilation slots for the air supply to and from the module cooling fan and these must be kept clear of obstructions at all times. The rear panel of the unit incorporates the feed gas inlet connector and the ozone outlet connector as well as the mains electrical connection socket. The front panel is fitted with an illuminated ozone on-off switch, fault indicator, variable output controller and feedfas flowmeter. 

\section{Operation}


\NP In order to operate the ozone generator there are several procedures that must be followed.

\begin{itemize}
  \item 1. Electrically connect the unit to the mains supply utilising the cable supply lead supplied with the unit.
  \item 2. Start feeding gas through the generator and set the generator to the required flowrate.
  \item 3. Depress the main ON-OFF switch on the generator front panel which will illuminate indicating that ozone gas is being produced.
  \item 4. Set the variable control knob to the output required by utilising the output graphs included in this manual.
  \item 5. The unit takes ten minutes initially to reach its normal operating temperature and output. 
  \item NOTE: The red FAULT indicator will illuminate brifely after pressing the main ON-OFF switch due to the fractional delay in the power board relays latching.
\end{itemize}

\section{Errors}

\NP In the event of the red FAULT light illuminating, this indicates that there is a fault in the system which could be the result of the following problems:

\begin{itemize}
  \item CAUSE 1: The ozone module or power board have developed a short-circuit
  \item ACTION: Check the main fuses on the power board. Check the HT transformer and conncector cable for signs of arcing. 
  \item CAUSE 2: The thermostat on the ozone module cooling shroud has activated due to higher than normal operating conditions.
  \item ACTION: Check that the cooling fan is operaional. Check the unit is not being subkected to ambient temperatures above 40 degrees Celsius.
  \item CAUSE 3: The thermostat protection on the power board has activated due to higher than normal operating conditions. 
  \item ACTION: Switch off the ON-OFF switch and allow the unit to cool donw for fifteen minutes then restart the unit. If the unit starts properly, this indicates that the internal temperature of the enclousure is overheating. 
\end{itemize}

\section{Interferences}

\section{Health and Safety}

\subsection{Safety and Personnnel Protective Equipment}


\section{Personnel \& Training Responsibilities}

Researchers training to use the Eosense chambers and Picarro analyzer include the following components: 



Researchers using this SOP should be trained for the following SOPs:

\begin{itemize}
  \item SOP03 Field Work
  \item SOP04 Electrical Power in the Field
\end{itemize}

\section{Required Materials}

\subsection{Item 1 w/catalog number!}
\subsection{Item 2}

\section{Estimated Time}

\NP This will take XX minutes...

\section{Procedure}

\NP Prepare \dots

\NP

\section{References}

\NP APHA, AWWA. WEF. (2012) Standard Methods for examination of water and wastewater. 22nd American Public Health Association (Eds.). Washington. 1360 pp. (2014).

\end{document}
