%SOP Template 
% Version 02 Created SOP



\documentclass[12pt]{../SOP4_alpha}\usepackage[]{graphicx}\usepackage[]{color}
%% maxwidth is the original width if it is less than linewidth
%% otherwise use linewidth (to make sure the graphics do not exceed the margin)
\makeatletter
\def\maxwidth{ %
  \ifdim\Gin@nat@width>\linewidth
    \linewidth
  \else
    \Gin@nat@width
  \fi
}
\makeatother

\definecolor{fgcolor}{rgb}{0.345, 0.345, 0.345}
\newcommand{\hlnum}[1]{\textcolor[rgb]{0.686,0.059,0.569}{#1}}%
\newcommand{\hlstr}[1]{\textcolor[rgb]{0.192,0.494,0.8}{#1}}%
\newcommand{\hlcom}[1]{\textcolor[rgb]{0.678,0.584,0.686}{\textit{#1}}}%
\newcommand{\hlopt}[1]{\textcolor[rgb]{0,0,0}{#1}}%
\newcommand{\hlstd}[1]{\textcolor[rgb]{0.345,0.345,0.345}{#1}}%
\newcommand{\hlkwa}[1]{\textcolor[rgb]{0.161,0.373,0.58}{\textbf{#1}}}%
\newcommand{\hlkwb}[1]{\textcolor[rgb]{0.69,0.353,0.396}{#1}}%
\newcommand{\hlkwc}[1]{\textcolor[rgb]{0.333,0.667,0.333}{#1}}%
\newcommand{\hlkwd}[1]{\textcolor[rgb]{0.737,0.353,0.396}{\textbf{#1}}}%
\let\hlipl\hlkwb

\usepackage{framed}
\makeatletter
\newenvironment{kframe}{%
 \def\at@end@of@kframe{}%
 \ifinner\ifhmode%
  \def\at@end@of@kframe{\end{minipage}}%
  \begin{minipage}{\columnwidth}%
 \fi\fi%
 \def\FrameCommand##1{\hskip\@totalleftmargin \hskip-\fboxsep
 \colorbox{shadecolor}{##1}\hskip-\fboxsep
     % There is no \\@totalrightmargin, so:
     \hskip-\linewidth \hskip-\@totalleftmargin \hskip\columnwidth}%
 \MakeFramed {\advance\hsize-\width
   \@totalleftmargin\z@ \linewidth\hsize
   \@setminipage}}%
 {\par\unskip\endMakeFramed%
 \at@end@of@kframe}
\makeatother

\definecolor{shadecolor}{rgb}{.97, .97, .97}
\definecolor{messagecolor}{rgb}{0, 0, 0}
\definecolor{warningcolor}{rgb}{1, 0, 1}
\definecolor{errorcolor}{rgb}{1, 0, 0}
\newenvironment{knitrout}{}{} % an empty environment to be redefined in TeX

\usepackage{alltt}

\usepackage[english]{babel}

\title{Becoming a IRMS User/User Handbook}
\date{2/12/2018}
\author{Marc Los Huertos and Kyle McCarty}
\approved{TBD}
\ReviseDate{\today}
\SOPno{75A v0.2}
\IfFileExists{upquote.sty}{\usepackage{upquote}}{}
\begin{document}


\maketitle

\section{Scope and Application}

\NP The scope of this standard operating procedure (SOP) defines who can use the isotope-ratio mass spectrometer (\hyperref[IRMS]{IRMS}) and its peripherals, the training required to become a user and/or super-user, and to be used as an informational guide on the various topics related to the IRMS.

\NP The applications of this SOP are for researchers to learn how to use the Oxtoby Isotope Lab IRMS. Using the IRMS requires attention to detail and the users must be qualified to use the instruments. The lab manager does not have the time or capacity to run samples for researchers, but can train users to run their samples. Completing this SOP is the first step toward becoming a user or super-user.

\section{Summary of Training}

\NP Since the Oxtoby lab is managed by a 1/2 manager, it's important the users are able to run the instruments independently -- but they are expensive and very involved, so we need to ensure that users are not only qualified but also \emph{confident} in operating the instruments.

\NP Training will include background reading of relevant topics, observation of an already trained user, supervised experience running the instrumentation, and finally a start-to-finish experience that either the technician or specfic Super-Users will, with their discretion, determine a trainee a user.

\NP We do understand that each user will have their own particular need from the IRMS and it may not be relevant to know every kind of analysis, peripheral, part, etc. The lab technician and other super-users will do their best to create a compromise between the required relevant training and its associated time commitment for the trainees.

\newpage

\tableofcontents


\newpage
\section{Acknowledgements}

The laboratory was funded by the Moore Foundation and was dedicated by the college on Feb. 26, 2018. Martina Ebert spearheaded the proposal based on David Oxtoby's relationship with the Moore Foundation's Board of Directors/Trustees?. The laboratory construction was approved by the Dean of the College, Audrey Bilger, President Oxtoby, and the Treasurer Karen Sission. The consturction was developed and managed by Brian Faber and overseen by Bob Robertson.

\section{Estimated Time for Certification}

\NP Estimated time to become a user requires approximately 6 hours (give or take) of observation, training, and minimally supervised runs. This does not include the time required to read relevant reading material such as manuals and standard operating procedures.

\NP This time also depends on the extent of training a user will need. If a user is interested in only one application it will take less time to cover than if a user is interested in all of the possilbe applications.

\section{Procedure}

\NP Generally the procedure to become a certifed user will look like the following:

\begin{itemize}
  \item Complete the required reading.
  \item Read general background of how the IRMS works (\hyperref[sec:Background]{Section \ref*{sec:Background}}).
  \item Observe other user(s) operate the IRMS and its peripherals
  \item Read application SOP?? (SOP??: Applications of the IRMS and its Peripherals)
\end{itemize}

\subsection{Required Reading} \label{subsec:Required Reading}

\NP Researchers using this SOP to become an IRMS user should also be trained with the following SOPs:

\begin{itemize}
  \item SOP01 Laboratory Safety
  \item SOP?? Applications of the IRMS and its Peripherals
%  \item SOP02 Field Safety
\end{itemize}

\NP \textbf{Super-users} should be trained with this SOP as well as the following SOPs:

\begin{itemize}
  \item SOP01 Laboratory Safety
  \item SOP?? Application of the IRMS and its Peripherals
  \item SOP?? For Super-Users: How to perform periodic maintenance on the IRMS and its peripherals. 
\end{itemize}

\section{Laboratory Policies}

\NP In order to be able to run the IRMS and its associated peripherals, the person must be a certified user, i.e. completed the prerequisite training to become a User or Super-user. 

\NP The Oxtoby Laboratory is not a ``lab service'' and the manager, or others associated with the lab, cannot be used to run samples. The Manager will prepare the IRMS to ensure the the proper methods are working properly. However, it is important to note that the Manager will not prepare samples, create run sequences, oversee sequence runs, or conduct data reduction processes. 

\NP Research is a time commitment. Thus, users must be prepared to dedicate time to prepare and run their own samples in a timely fashion and respect other users of the lab.

\NP Instrument time is first-come first-serve, with the exception that the principle investigators (P.I.s) have priority.

\NP \textbf{Expect to be delayed} if you are immediately signed up after someone else. Unforseen events can arise and cause delays. Things happen. Be understanding and cautious here. Plan your time efficiently.

\NP Lab Access will generally be between the regular hours of 8am to 5pm unless the lab technician or Dr. Marc Los Huertos is within the department. Schedule may vary so it is best to contact the technician \href{mailto:kyle.mccarty@pomona.edu}{(kyle.mccarty@pomona.edu)} to either schedule time or create some sort of arrangement so you can gain access to the laboratory.

\NP The Oxtoby Isotope Laboratory relies on an ``in-kind payment'' approach to fund its operations. Although there is no per sample cost, per se, the laboratory expects consumables and gases to be replenished by users/super users relative to their use. If you happen to have certain consumables that you would like to use (instead of the lab's) feel free to do so. These items may include tin/silver capsules, cell wells, weighing paper, gloves, standards, reagents, etc.

\NP \textbf{CLEAN UP AFTER YOURSELF!} Clean up your Kimwipes, weighing papers, paper towels, note papers, etc. Organize your samples and personal items regardless if you are in or out of the lab!

\NP \textbf{PUSH IN YOUR CHAIRS!} Keep the laboratory as tidy as possible. We don't want to have any needless obstructions.

\section{Health and Safety Risks}

\NP Pressurized, reactive, and poisenous gases - Hydrogen (H$_2$), Oxygen (O$_2$), Carbon Monoxide (CO$_2$), Sulfur Dioxide (SO$_2$)

\NP Reagent and Acid Handling - 100\% Phosphoric acid for carbonate analysis, quartz wool for column packing, etc. 

\NP Risk of Burns (hot reactor and/or ash finger handling)

\NP Puncture/Cut related wounds from syringes, sharp objects, etc.

\section{Personnel \& Training Responsibilities/Requirements}

\NP Users will be held to high professional standards and violation will forfeit your privilege to use the lab.

\NP Personal protective equipment (PPE) should be worn at all times. Safety is of the utmost priority.

\NP Training is required before time can be scheduled to use the IRMS and its peripherals. It is your responsibility to schedule time to use instrumentation. If you do not schedule time accordingly you will lose lab privellages.


\section{Required Materials and Apparati}

\subsection{Safety and Personal Protective Equipment (PPE)}

\NP \textbf{Lab Coat} - Please bring your own lab coat to wear. We only have a few coats (if you do not have one) that can be lent out.

\NP \textbf{Safety Glasses} - Please bring your own safety glasses as well. We only have a few glasses to lend out.

\NP \textbf{Gloves} - Various types and sizes of gloves can be found in the lab. Feel free to take these as needed, and also feel free to donate gloves!

\subsection{Consumables}

\NP Supplying your own reagents, standards, and other items is encouraged. See the following section for a list of \hyperref[subsec:Consumables and Reagents for Flash EA]{reagents} (\ref{subsec:Consumables and Reagents for Flash EA}) and \hyperref[subsec:Standard Reference Materials]{standards} (\ref{subsec:Standard Reference Materials}) that the laboratory will generally have on hand. These reagents and standards can be used if needed, but you are expected to replace what you use.

\NP Other items such as gloves, Kimwipes, weighing paper, and things of that nature are supplied by the laboratory. Feel free to donate any items like this, though!

\NP It is also encouraged to bring your own consumeables appropriate for what peripheral you will be using. For example, if you are doing carbonate analysis try to bring your own supply of vials, caps, septa, needles, and syringes. If you are going to be using the EA or TC/EA, supplying your own reagents, reactor tubes, tin or silver capsules, cell wells, etc. will be encouraged.

\subsection{Non-consumables}

\NP Generally, you will not need to bring non-consumable items like tweezers, spatulas, scissors, etc. but it is a good idea to supply your own cell well and sample holders.

\NP Please bring your own test tubes and \textbf{bring your own test tube racks!}

\NP If you are really worried about sample contamination, just bring your own items.

\section{Background} \label{sec:Background}

\subsection{Gas Source Isotope Ratio Mass Spectrometry}

\NP Isotope ratio mass spectrometer instruments are designed to measure differences in the abundances of isotopes, usually of the lighter elements. Our particular instrument, the ThermoFisher Delta V Isotope Ratio Mass Spectrometer, can measure $^{2}$H/$^{1}$H, $^{13}$C/$^{12}$C, $^{15}$N/$^{14}$N, $^{18}$O/$^{16}$O, and $^{34}$S/$^{32}$S in a variety of types of analyses.
\\
\\
Prior to the determination of isotope ratios, the samples of interest are converted into simple gases such as H$_2$, CO, N$_2$, CO$_2$ and SO$_2$. The IRMS then measures these gases with their corresponding isotope ratios. The IRMS doesn't actually delineate between isotope species and can only tell the difference between ions by their mass to charge (m/z) ratios. For example, when the instrument is analyzing for $\delta$$^{15}$N it measures ions at mass to charge ratios of 28, 29, and 30. Technically, nitrogen has isotopes at all of  those m/z ratios, but CO exists at m/z 28 as well. The measured ratios from each analysis is then compared to the appropriate reference gases, and in turn, to verified standards in order to determine their delta values.\\
\\
The two most common gas source IRMS configurations are continuous flow (CF-IRMS) and dual-inlet IRMS (DI-IRMS). Our lab has a continuous flow system and these systems usually utilize an elemental analyser of some kind. DI-IRMS is generally 10-fold more precise than CF-IRMS but requires more sample preparation, larger sample sizes, and less sample throughput.

\subsection{Continuous Flow IRMS Peripheral Interaction}

\NP Explain the interaction of the peripherals and IRMS here.

\NP Add system diagram here.

\begin{description}

\item[ConFlo IV]

\item[Flash EA]

\item[TC/EA]

\item[GasBench II]

\end{description}

\section{Gases, Reagents, and Standards}

\subsection{Reference/Standard Gases}

\subsubsection{Tank Farm 1 (West wall most northern)}

\NP Helium (He) - Tank pressure should be above 500psi and regulated at 50psi.

\NP Nitrogen (N$_2$)

\NP Carbon Dioxide (CO$_2$)

\NP Oxygen (O$_2$)

\subsubsection{Tank Farm 2 (West wall most southern)}

\NP Hydrogen (H)

\NP Carbon Monoxide (CO)

\NP Hydrogen and Helium (H and He)

\NP Hydrogen and Carbon Dioxide (H and CO$_2$)

\NP Sulfur Dioxide (SO$_2$)

\subsection{Consumables and Reagents for Flash EA} \label{subsec:Consumables and Reagents for Flash EA}

\NP There are multiple different setups and combinations for the Flash EA. Make sure to verify the reactor(s) is the proper one for your analysis. Also, you can save money on not going direct to Thermo for parts and reagents. A good alternative is \href{https://eaconsumables.com/}{EA Consumables}.

\newpage

\subsubsection{NC Determination (Dual Reactor Setup)}

\begin{table}[h]
\label{NC Dual Reactor item list}
\caption{NC Dual Reactor items}
\centering
\begin{tabular}{lccc} \hline
Item    & ThermoFisher P/N  & EA Consumables P/N \\ \hline\hline
Quartz Reactor (set of 2)   & 468 20070 & C1069 \\ 
Ash finger (quartz insert)    & N/A & C1098 \\ 
Silvered Cobaltous/Cobaltic oxide   & 338 24500 & B1120 \\ 
Chromium oxide    & 338 22900 & B1099 \\ 
Reduced copper wires    & 338 35300 & B1016 \\ 
Quartz wool   & 338 22200 & ??? \\ 
Magnesium Perchlorate (Anhydrone)   & 281 13102 & ??? \\ 
\end{tabular}
\end{table}

\begin{table}[h]
\label{column}
\caption{column}
\centering
\begin{tabular}{lccc} \hline
Analysis      & XX    & Copper  & \\ \hline\hline
CN            & Yes   & No      & Yes \\ \hline
\end{tabular}
\end{table}

\newpage

\subsubsection{NCS/Single Sulfur Determination}

\NP Sulfur determination requires the system to be retrofitted properly. It will require a dedicated reactor, dedicatd separation column for S analysis, special tubing to minimize water absorption, and special heating foil for the ConFlo and SO$_2$ refrence gas regulator.


\begin{table}[h]
\label{NCS/Single Sulfur Reactor item list}
\caption{NCS/Single Sulfur Reactor items}
\centering
\begin{tabular}{lccc} \hline
Item    & ThermoFisher P/N & EA Consumables P/N \\ \hline\hline
Quartz Reactor (set of 2)   & 468 20070 & C1069 \\ 
Ash finger (Qtz insert)    & N/A & C1098 \\ 
Quartz wool   & 338 22200 & ??? \\ 
Tungsten oxide    & 338 24600 & ???\\ 
Electrolytic copper   & 338 35314 & ???\\ 
Vanadium pentoxide   & 338 37510 & ???\\ 
Sulfinert separation column (2m)   & 128 9970 & ??? \\ 
\end{tabular}
\end{table}

\NP Reaction Column packing for NCS/Single Sulfur Determination (Table~\ref{column})

\begin{table}[h]
\label{column}
\caption{column}
\centering
\begin{tabular}{lccc} \hline
Analysis      & XX    & Copper  & \\ \hline\hline
CN            & Yes   & No      & Yes \\ \hline
\end{tabular}
\end{table}

\newpage

\subsection{Consumeables and Reagents for TC/EA} \label{subsec:Consumables and Reagents for TC/EA}

\subsubsection{H, O, and Dual Measurement (Silver Capsule $\leq$ 3mm)}

\begin{table}[h]
\label{H, O, and Dual Measurement (Silver Capsule 3mm}
\caption{H, O, and Dual Measurement (Silver Capsule $\leq$ 3mm)}
\centering
\begin{tabular}{lccc} \hline
Item    & ThermoFisher P/N  & EA Consumables P/N \\ \hline\hline
Ceramic tube   & 1121320 & ??? \\ 
Graphite tube     & 1121351 & ??? \\ 
Graphite crucible     & 1117332 & ??? \\ 
Glassy carbon reactor     & 1121310 & ??? \\ 
Glassy carbon granulate     & 1117400 & ??? \\ 
Silver wool     & 1117430 & ??? \\ 
Quartz wool     & 1087770 & ??? \\
(Optional) Reactor ready for use     & 1127500 & \\
\end{tabular}
\end{table}

\begin{table}[h]
\label{column}
\caption{column}
\centering
\begin{tabular}{lccc} \hline
Analysis      & XX    & Copper  & \\ \hline\hline
CN            & Yes   & No      & Yes \\ \hline
\end{tabular}
\end{table}


%($^{2}$H/$^{1}$H)
%($^{18}$O/$^{16}$O)

\newpage

\subsubsection{H, O, and Dual Measurement (Silver Capsule 6mm $\geq$ diameter $\geq$ 3mm )}

\begin{table}[h]
\label{H, O, and Dual Measurement item list}
\caption{H, O, and Dual Measurement items}
\centering
\begin{tabular}{lccc} \hline
Item    & ThermoFisher P/N  & EA Consumables P/N \\ \hline\hline
Ceramic tube   & 1121320 & ??? \\ 
Graphite crucible     & 1117332 & ??? \\ 
Glassy carbon reactor     & 1121310 & ??? \\ 
Glassy carbon granulate     & 1117400 & ??? \\ 
Silver wool     & 1117430 & ??? \\ 
Quartz wool     & 1087770 & ??? \\
(Optional) Reactor ready for use     & 1127500 & \\
\end{tabular}
\end{table}

\begin{table}[h]
\label{column}
\caption{column}
\centering
\begin{tabular}{lccc} \hline
Analysis      & XX    & Copper  & \\ \hline\hline
CN            & Yes   & No      & Yes \\ \hline
\end{tabular}
\end{table}

\newpage

\subsubsection{H, O, and Dual Measurement (Liquid Injection)}

\begin{table}[h]
\label{H, O, and Dual Measurement item list}
\caption{H, O, and Dual Measurement items}
\centering
\begin{tabular}{lccc} \hline
Item    & ThermoFisher P/N  & EA Consumables P/N \\ \hline\hline
Ceramic tube   & 1121320 & ??? \\ 
Graphite tube     & 1132080 & ??? \\ 
Glassy carbon reactor     & 1121310 & ??? \\ 
Glassy carbon granulate     & 1117400 & ??? \\ 
Quartz wool     & 1087770 & ??? \\
\end{tabular}
\end{table}

\begin{table}[h]
\label{column}
\caption{column}
\centering
\begin{tabular}{lccc} \hline
Analysis      & XX    & Copper  & \\ \hline\hline
CN            & Yes   & No      & Yes \\ \hline
\end{tabular}
\end{table}

\newpage

\subsection{Standard Reference Materials} \label{subsec:Standard Reference Materials}

\NP As stated before, the laboratory has some standards but they aren't to be used willy nilly without the blessing of a super-user, technician, or P.I. You should already know what standards you will need for your analysis before approaching the lab. The list below is just for reference of what exists in the lab and their certified values.

\begin{table}[h]
\label{Isotopic Standard Reference Materials}
\caption{Isotopic Standard Reference Materials}
\centering
\begin{tabular}{lccc} \hline
Reference Standard  &$\delta$2H \relsize{-2}(VSMOW-SLAP) &$\delta$13C \relsize{-2}(VPDB-LSVEC)  &$\delta$15N \relsize{-2}(Air-N2) \\ \hline\hline
LOCSS &-  &-26.66  &7.30 \\
HOCSS &-  &-26.27  &4.42 \\
Urea &- &-43.26  &-0.56\\
USGS24 &- &-16.05  &- \\
USGS40 &- &-26.39  &-4.52 \\
USGS41a &-  &36.55 &47.6\\
USGS57 &-91.5 &-  &-\\
USGS58 &-28.4  &-  &-\\
USGS61 &96.9  &-35.05 &-2.87\\
USGS62 &-156.1  &-14.79 &20.2\\

\end{tabular}
\end{table}

\section{Time Management}

\subsection{Overview}

\NP Using the IRMS system \textbf{TAKES A LOT OF TIME} so make sure you plan your time accordingly. Remember to be attentive, patient, and persistent. Things do happen and things do go wrong, but make sure you do not end up ruining the instrument because you are running a tight schedule. Set aside multiple, relatively uninterrupted, days when you can sit down and take the time to think through the process.

\NP Plan ahead to make sure the system, as a whole, gets baked out properly. You want to make sure you don't have any residual compounds or gases lingering or being produced before you proceed.

\NP In addition, you'll spend a decent amount of time verifying that background levels are within reason. 

\NP Keep in mind zero-enrichment tests (on/offs) need to be preformed to verify instrument is functioning properly. This process is usually underestimated in terms of the time it takes. Not only is this process variable, but the instrument can fall out of specification if it sits idle too long (even after initially meeting its specifications). Generally, one single on-off sequence takes approximately 10 minutes, and you almost always need to do multiple on-off sequences before running an analysis.

\NP Analysis time varies depending on method, analysis, and peripheral. See each following section on different types of methods that already exist and their approximate time per analysis.

\subsection{Flash EA}

\NP Determine the number of samples will be analyzed, how many accompanying standards will be needed (depending on your data correction scheme), and blanks. Keep in mind the autosampler carousels have 32 wells each. You can stack all three carousels for a total of 94 samples.

\subsubsection{Nitrogen ($\delta$$^{15}$N)}

\NP Takes approximately 3 mintues and 45 seconds plus an additional minute or so for peak centering and magnet switching.

\subsubsection{Carbon ($\delta$$^{13}$C)}

\NP Takes approximately 5 minutes plus an additional minute or so for peak centering and magnet switching.

\subsubsection{NC dual method}

\NP Takes approximately 7 mintues plus and additional minute or so for peak centering and magnet switching.

\subsubsection{NCS triple analysis} 

\NP Takes approximately 10 minutes and 45 seconds plus an additional minute or so for peak centering and magnet switching.

\subsection{GasBench II}

\NP The GasBench is a low flow peripheral so it generally doesn't have much in terms of background signatures. Being a low-flow peripheral, it can generally pass zero-enrichment tests a lot quicker than a high-flow peripheral.

\NP There is the added complexity of using the autosampler (PAL system) with the GasBench. It is a good idea to decide whether you will be using the autosampler and IRMS to flush vials online or to do the flushing stage offline. There is considerable time discrepency here as you cannot be analyzing samples on the IRMS while automating the flushing stage. You can, however, flush vials offline while analyzing another batch of samples.

\NP For most methods, the GasBench samples slightly differently than say the Flash EA does. The GasBench samples a single vial multiple times, 11 to be exact. This makes the analysis time a little longer than you would initially anticipate.

\subsubsection{Carbonates}

\NP The analysis of a carbonate uses the CO2\textunderscore 630 method; this takes approximately 10 mintues and 30 seconds plus an additional minute or so for peak centering and magnet switching.

\subsubsection{Dissolved Inorganic Carbon (DIC)}

\subsubsection{Breath Gas Analysis}

\subsubsection{CO$_2$ in Atmospheric Concentrations}

\subsubsection{Water Equilibration ($^{18}$O/$^{16}$O)}

\subsubsection{Water Equilibration ($^2$H/$^1$H)}

\subsection{TC/EA}

\NP The TC/EA will function similar to the Flash EA in terms of the time it takes for backgrounds to get to reasonable levels. The TC/EA also needs to do an H3 Factor test before analysis. This test takes about 8 or so minutes to complete.

\NP The TC/EA consumes a lot more helium than the Flash EA does, as it does not have a helium-saving feature. Please budget your time carefully and work as efficiently as possible!

\subsubsection{Hydrogen Measurement ($\delta$$^{2}$H)}

\NP Takes approximately 3 mintues and 35 seconds plus an additional minute or so for peak centering and magnet switching.

\subsubsection{Oxygen Measurement ($\delta$$^{18}$O)}

\NP Takes approximately \textbf{X} mintues plus an additional minute or so for peak centering and magnet switching.

\subsubsection{Dual Measurement}

\NP Takes approximately 6 mintues plus an additional minute or so for peak centering and magnet switching.


\section{Pre-Analysis Sequence and QC/QA Criteria}

\subsection{Mass Spectrometer Checks}

\NP The easiest place to begin is to make sure the IRMS is reading back and operating normally.

\NP Visually verify that there are no red or yellow lights on the front panel of the IRMS and that the vacuum is within acceptable range.

\NP Double check to make sure you have enough helium and reference gases for your analysis and that the gas flows are within specification.

\NP If the source is already on, verify the high voltage (HV), box, and trap values are reading back.

\subsection{Tuning}

\NP An IRMS can be generally tuned in two different ways. It is usually said that the IRMS is tuned for ``linearity,'' where there are consistent ion ratios over a wide range of signal intensities, or ``sensitivity,'' which is intended for obtaining the maximum signal intensity.

\NP Since our instrument is not set up with a dual inlet system, the approach of tuning for "linearity" should be used.

\subsection{Pre-Analysis Tests (analysis dependant)}

\NP The following tests should be done based on what kind of analysis you are doing and what peripheral you are using.

\NP If you analyze for $\delta$$^2$H, you must determing a H$_3$$^+$ factor.

\NP Only solid samples need blank determinations.

\NP All other kinds of analyses will need background, zero-enrichment, and QC/QA tests.

\subsubsection{Backgrounds}

\NP Backgrounds are tested to make sure the residual gases in the source do not contribute to a significant amount of signal during analysis.

\NP Backgrounds also give insight into the condition of the residual gases in the ion source. They help to determine leaks and the degree of which a residual gas may contribute to a signal.

\begin{table}[h]
\label{Backgrounds}
\caption{Backgrounds and Associated Causes}
\centering
\begin{tabular}{lccc} \hline
m/z & Mol Species & Problem or possible cause \\ \hline\hline
2 & He$^2$$^+$ & \\
18 & H$_2$O$^+$ & \\
28 & N$_2$$^+$ & \\
40 & Ar$^+$ & \\
44 & CO$_2$$^+$ & \\
\end{tabular}
\end{table}


\subsubsection{Zero-Enrichment Test}

\subsubsection{H3$^+$ Factor}

\subsubsection{Blank Determination}

\subsection{Quality Assurance/Quality Control}

\section{Troubleshooting}

\NP Question \\
Answer

\section{Definitions}

\begin{description}

\item[ConfloIV] A unique peripheral to the IRMS as it is the hub for all plumbed gas lines from each of the other peripherals, reference gases, and the IRMS itself. The Conflo also conducts the proper dilutions of sample and reference gases that is needed for them to fall within the working range of the IRMS.

\item[Delta V IRMS] \label{IRMS} The Oxtoby Lab's Isotope Ratio Mass Spectrometer, model Delta V manufactured by ThermoFisher Scientific. 

\item[Flash IRMS EA] Also know as the ``Flash'' or ``Flash EA." It is an elemental analyzer and one of the three peripherals for the Delta V IRMS. It combusts a multitude of sample types to produce varying gases for the IRMS to analyze. 

\item[Gasbench II] Is another periphepheral for the IRMS that samples (usually in conjunction with an autosampler), treats, and transports sample gases from sealed vials.

\item[H3 Factor]

\item[Linearity] In simplest terms, linearity is the relationship between measured $\delta$ values and signal amplitude (or signal area). There are a multitude of reasons for fractionation to occur in the ion source and, experimentally, there is a functional dependence between $\delta$ values and peak amplitude (or peak area). Therefore, the distribution of results are distributed along a line with a small slope. This slope is the linearity (peak amplitude) correction factor. 

\item[Student Researcher] Is generally going to be a student who either does not feel confident in becoming a user or simply doesn't have the time to invest in it. Although, a student researcher can conduct sample weighing, data reduction, and sequence creation.

\item[Super-User] \label{Super-User} A staff or faculty memember who is qualified to run and perform minor maintanence on the IRMS, including, but not necessarily limted to, gas replacement, reactor exchange, needle exchange, PAL system programming, etc.  

\item[TC/EA] Thermal Conversion/Elemental Analyzer; similar to the Flash EA but instead uses pyrolysis and much higher temperatures (approximately 1450*C) to convert sample material into gases analyzed by the IRMS.

\item[User] A student, staff, or faculty member who has qualified to prepare samples and run the IRMS without supervsion.

\item[Zero-Enrichment Test]

\end{description}


\section{References}

\NP APHA, AWWA. WEF. (2012) Standard Methods for examination of water and wastewater. 22nd American Public Health Association (Eds.). Washington. 1360 pp. (2014).

\end{document}
