%SOP Template 
% Version 02 Added revision date
% Version 03 Added TOC and acknowledgements
%           New SOP3_alpha.cls


\documentclass[12pt]{../SOP3_alpha}
\usepackage[english]{babel}
\usepackage{blindtext}
\usepackage{lipsum}

%\documentclass{article}

%\documentclass[12pt]{~/github/SOPs/SOP_Template/SOP}

\title{DNA Isolation from Soil --MoBio PowerSoil DNA Kit}
\date{8/15/2016}
\author{Marc Los Huertos}
\approved{TBD}
\ReviseDate{\today}
\SOPno{39 v.01}

\usepackage{Sweave}
\begin{document}
\Sconcordance{concordance:Isolating_Soil_DNA_v01.tex:Isolating_Soil_DNA_v01.Rnw:%
1 271 1 50 0}


\maketitle

\section{DNA Extraction }

\NP The scope of this SOP is train researchers to become familiar with the steps and procedures involved in extracting genomic DNA from plant material, and in particular, algal species. With the use of equipment and materials provided in the Nucleospin Plant II Kit, DNA from plant samples can be successfully extracted by following proper protocol, safety precautions, and by paying close attention to the detailed instructions outlined in this handout, alongside the Professor. 

\NP The applications of this SOP are for various types of plant samples. As long as samples can be homogenzied, this procedure for DNA extraction is applicable. 

\section{Summary of Method}

Soil samples are homogenized by mechanical treatment or collected in a manner that does not require additional treatment (i.e. samples suspended in water/solvent). 

The PowerSoil® DNA Isolation Kit is effective at removing PCR inhibitors from even
the most difficult soil types. Environmental samples are added to a bead beating
tube for rapid and thorough homogenization. Cell lysis occurs by mechanical and
chemical methods. Total genomic DNA is captured on a silica membrane in a spin
column format. DNA is then washed and eluted from the membrane. DNA is then
ready for PCR analysis and other downstream applications. 

\tableofcontents

\newpage

\section{Acknowledgements}

\NP This SOP was originally written by Aparna... to extract DNA from surface waters. Initial test of the methods were done by Alejandro Guerrero in the summer of 2016. 

\section{Definitions}

\NP Term1: is... All reagants and equipment are self explanatory in lab. 

\section{Biases and Interferences}

\NP The extraction process is supposed to isolate DNA from other celluular or other organic materials. However, proteins and phenols...  

\NP When meausuring the DNA yield, these compounds may interfere with the measurements and thus, suggest you have more DNA that was extracted in reality. 

\NP Mesuring the absorbance ratios of several wavelength gives a reasonble estimate of DNA purity. See the QA/QC section to measure these contaminants.

\begin{description}
  \item[A260/A280 Ratio] Nucleic acids, DNA and RNA, absorb at 260nm. For a pure sample, a well defined peak (no shoulders or wiggles) at 260nm is expected.Several factors, however, can influence the accuracy of the 260/280 and 260/230 ratios. Readings from very dilute samples will have very little difference between the absorbance at 260 and 280nm leading to inaccurate ratios.  The type(s) of protein present will also have an effect.  Absorbance in the UV range by proteins is primarily the result of aromatic ring structures. Phenol and other contaminants can also absorb at 280 nm and can affect the ratio calculation. Phenol absorbs with a peak at 270nm. \textbf{Nucleic acid preparations uncontaminated by phenol should have an 260/280 ratio of around 1.8. } The pH of the solution can also affect the 260/280 ratio, with acidic solutions having a lower ratio of up to 0.2–0.3 and alkaline solutions having an increased ratio by a similar amount.
  \item[A260/A230 Ratio] Pure RNA has an A260/A280 ratio of 2.0, therefore if a DNA sample has an 260/280 ratio of greater than 1.8 this could suggest RNA contamination. The 260/280 ratio is a secondary measure of nucleic acid purity. This ratio for pure samples are often higher than the respective 260/280 ratio values. Strong absorbance around 230nm can indicate that organic compounds or chaotropic salts are present in the purified DNA.  A ratio of 260nm to 230nm can help evaluate the level of salt carryover in the purified DNA. The lower the ratio, the greater the amount of salt present. A\textbf{s a guideline, the 260/230 ratio should be greater than 1.5, ideally close to 1.8. } Urea, EDTA, carbohydrates and phenolate ions all have absorbance near 230nm. A reading at 320nm will indicate if there is turbidity in the solution, another indication of possible contamination.  

\end{description}



\section{Health and Safety}

\NP The risks involved with using this kit are mainly related to the chemicals that are in use. As listed below, safety precautions should be taken at all times, but especially when handling hazardous reagants. While following safety protocol, it is advised that the materials and equipment are kept organized to avoid contamination and thus yield good results. 

Please wear gloves when using this product. Avoid all skin contact with
reagents in this kit. In case of contact, wash thoroughly with water. Do not
ingest. See Material Safety Data Sheets for emergency procedures in case
of accidental ingestion or contact. All SDS information is available upon
request (760-929-9911) or on our web site at www.mobio.com. Reagents
labeled flammable should be kept away from open flames and sparks.
WARNING: Solution C5 contains ethanol. It is flammable. Do not use bleach to
clean the inside of the PowerVac™ Manifold or to rinse the PowerVac™ Mini Spin
Filter Adapters when attached to the manifold.

\subsection {Safety and Personnnel Protective Equipment}

Always wear appropriate lab safety equipment, including safety goggles, lab coat, close-toed shoes, long pants, and gloves. 
CAUTION: PC and PW1 contain guanidine hydrochloride, ethanol, and isopropanol, beware of these chemicals coming into contact with the skin and especially the eyes. Also keep away from heat (highly flammable liquid and vapours)


\section{Personnel \& Training Responsibilities}

\NP Researchers training is required before this the procedures in this method can be used... 

\NP Researchers using this SOP should be trained for the following SOPs:

\begin{itemize*}
  \item SOP01 Laboratory Safety
  \item SOP02 Field Safety
  \item SOP03 Handling of Hazardous Materials
  \item SOP12 Using Hot Plates and Dry Baths
  \item SOP14 Microcentrifuge
  \item SOP09 Using Balances, Pippettes, and Glassware
  \item SOP16 Using Laboratory Refrigerators and Freezers
\end{itemize*}

\section{Required Materials}

\subsection*{Equipment}

\NP Bead homogenizer

\NP Microcentrifuge with rotor capable of reaching 4500g

\NP Piettors 2 $\mu$L - 750 $\mu$L

\NP Vortex

\NP thermal heating bloock or water bath for incubation

\NP elution, mortar and pestle (if necessary for homogenization)

\subsection*{Reagents}

\NP 96-100\% Ethanol 

\subsection*{Consumables}

\NP 1.5mL microcentrifuge tubes, 

\NP Disposable pippet tips

\section{Estimated Time}

\NP This procedure requires approximately 5-6 hours depending on the number of samples,and the time it takes to homogenize starting material and complete other preliminary steps. 

\section{Sample Collection, Preservation, and Storage}

\NP Before the extraction process, soils should be either freshly collected or frozen... MORE here...

\NP After extraction, the DNA can be stored in the -80\degree, but be sure to follow the freezer SOP to appropriate...

\section{Procedure}

\subsection*{Preliminary Steps} 
 
\NP Mechanical treatment of soil samples can be done by grinding the material with a mortar and pestle in the presence of liquid nitrogen, without letting the sample thaw any time during this procedure. The mortar and pestle and spatula to remove the sample must be precooled before grinding the sample into a fine powder.

Please wear gloves at all times

\NP To the PowerBead Tubes provided, add 0.25 grams of soil sample.

What's happening: After your sample has been loaded into the PowerBead Tube, the next step is a homogenization and lysis procedure. The PowerBead
Tube contains a buffer that will (a) help disperse the soil particles, (b) begin to dissolve humic acids and (c) protect nucleic acids from degradation.

\NP Gently vortex to mix.

What's happening: Gentle vortexing mixes the components in the PowerBead Tube and begins to disperse the sample in the PowerBead Solution.

\NP Check Solution C1. If Solution C1 is precipitated, heat solution to 60 \degree C until the precipitate has dissolved before use.

What's happening: Solution C1 contains SDS and other disruption agents required for complete cell lysis. In addition to aiding in cell lysis, SDS is an anionic detergent that breaks down fatty acids and lipids associated with the cell membrane of several organisms. If it gets cold, it will form a white precipitate in the bottle. Heating to 60 \degree C will dissolve the SDS and will not harm the SDS or the other disruption agents. Solution C1 can be used while it is still warm.

\NP Add 60 $\mu$l of Solution C1 and invert several times or vortex briefly.

\NP Secure PowerBead Tubes horizontally using the MO BIO Vortex Adapter tube holder for the vortex (MO BIO Catalog \# 13000-V1-24) or secure tubes
horizontally on a flat-bed vortex pad with tape. Vortex at maximum speed for 10 minutes.

Note: If you are using the 24 place Vortex Adapter for more than 12 preps, increase the vortex time by 5-10 minutes. 

Note: The vortexing step is critical for complete homogenization and cell lysis. Cells are lysed by a combination of chemical agents from steps 1-4 and mechanical shaking introduced at this step. By randomly shaking the beads in the presence
of disruption agents, collision of the beads with microbial cells will cause the cells to break open.

Note

What's happening: The MO BIO Vortex Adapter is designed to be a simple platform to facilitate keeping the tubes tightly attached to the vortex. It should be noted that although you can attach tubes with tape, often the tape becomes loose and not all tubes will shake evenly or efficiently. This may lead to inconsistent results or lower yields. Therefore, the use of the MO BIO
Vortex Adapter is a highly recommended and cost effective way to obtain maximum DNA yields.

\NP Make sure the PowerBead Tubes rotate freely in your centrifuge without rubbing. Centrifuge tubes at 10,000 x g for 30 seconds at room temperature.

CAUTION: Be sure not to exceed 10,000 x g or tubes may break.

\NP Transfer the supernatant to a clean 2 ml Collection Tube (provided). Expect between 400 to 500 $\mu$l of supernatant at this step. The exact recovered volume depends on the absorbency of
your starting material and is not critical for the procedure to be effective. The supernatant may be dark in appearance and still contain some soil particles. The presence of carry over soil or a dark color in the mixture is expected in many
soil types at this step. Subsequent steps in the protocol will remove both carry over soil and coloration of the mixture.

Note

\NP Add 250 $\mu$l of Solution C2 and vortex for 5 seconds. Incubate at 4\degree C for 5
minutes.

What's happening: Solution C2 is patented Inhibitor Removal Technology® (IRT). It contains a reagent to precipitate non-DNA organic and inorganic material including humic substances, cell debris, and proteins. It is important to remove contaminating organic and inorganic matter that may reduce
DNA purity and inhibit downstream DNA applications.

\NP Centrifuge the tubes at room temperature for 1 minute at 10,000 x g.

\NP Avoiding the pellet, transfer up to 600 $\mu$l of supernatant to a clean 2 ml Collection Tube (provided).

What's happening: The pellet at this point contains non-DNA organic and inorganic material including humic acid, cell debris, and proteins. For the
best DNA yields, and quality, avoid transferring any of the pellet.

\NP Add 200 μl of Solution C3 and vortex briefly. Incubate at 4 \degree C for 5 minutes.

What's happening: Solution C3 is patented Inhibitor Removal Technology® (IRT) and is a second reagent to precipitate additional non-DNA organic
and inorganic material including humic acid, cell debris, and proteins. It is important to remove contaminating organic and inorganic matter that may
reduce DNA purity and inhibit downstream DNA applications.

\NP Centrifuge the tubes at room temperature for 1 minute at 10,000 x g.

\NP Transfer up to 750 μl of supernatant to a clean 2 ml Collection Tube (provided).

What's happening: The pellet at this point contains additional non-DNA organic and inorganic material including humic acid, cell debris, and proteins. For the best DNA yields, and quality, avoid transferring any of the pellet.

\NP Shake to mix Solution C4 before use. Add 1.2 ml of Solution C4 to the supernatant (be careful solution doesn’t exceed rim of tube) and vortex for 5 seconds.

What's happening: Solution C4 is a high concentration salt solution. Since DNA binds tightly to silica at high salt concentrations, this will adjust the
DNA solution salt concentrations to allow binding of DNA, but not non-DNA organic and inorganic material that may still be present at low levels, to the Spin Filters.

\NP Load approximately 675 $\mu$l onto a Spin Filter and centrifuge at 10,000 x g for 1 minute at room temperature. Discard the flow through and add an additional 675 μl of supernatant to the Spin Filter and centrifuge at 10,000 x g for 1 minute at room temperature. Load the remaining supernatant onto the Spin Filter and centrifuge at 10,000 x g for 1 minute at room temperature. A total of three loads for each sample processed are required.

Note

What's happening: DNA is selectively bound to the silica membrane in the Spin Filter device in the high salt solution. Contaminants pass through the
filter membrane, leaving only DNA bound to the membrane.

\NP Add 500 $\mu$l of Solution C5 and centrifuge at room temperature for 30 seconds at 10,000 x g.

What's happening: Solution C5 is an ethanol based wash solution used to further clean the DNA that is bound to the silica filter membrane in the
Spin Filter. This wash solution removes residual salt, humic acid, and other contaminants while allowing the DNA to stay bound to the silica membrane.

\NP Discard the flow through from the 2 ml Collection Tube.

What's happening: This flow through fraction is just non-DNA organic and inorganic waste removed from the silica Spin Filter membrane by the ethanol wash solution.


NOTE: It is important to avoid any traces of the ethanol based wash solution.
Note

\NP Add 100 $\mu$l of Solution C6 to the center of the white filter membrane. Placing the Solution C6 (sterile elution buffer) in the center of the small white membrane will make sure the entire membrane is wetted. This will result in a more efficient and complete release of the DNA from the silica Spin Filter membrane. As Solution C6 (elution buffer) passes through the silica membrane, DNA that
was bound in the presence of high salt is selectively released by Solution C6 (10 mM Tris) which lacks salt. Alternatively, sterile DNA-Free PCR Grade Water may be used for DNA elution from the silica Spin Filter membrane at this step (MO BIO Catalog \# 17000-10). Solution C6 contains no EDTA. If DNA degradation is a concern, Sterile TE may also be used instead of Solution C6 for elution of DNA from the Spin Filter.

Note

\NP Centrifuge at room temperature for 30 seconds at 10,000 x g.

\NP Discard the Spin Filter. The DNA in the tube is now ready for any downstream application. No further steps are required. We recommend storing DNA frozen (-20\degree to -80 \degree C). Solution C6 does not contain any EDTA. To concentrate DNA see the Hints \& Troubleshooting Guide.

\section{QA/QC}

\subsection*{Analysis of DNA Yield and Purity}

\NP The Nanodrop...

\NP Open machine software and click on Nucleic Acid.

\NP Clean machine with Kimwipe before loading anything onto machine. 
\NP Pipette the Blank, whatever was used during elution (i.e. Elution Buffer), using a 2 $\mu$L pipettor onto the hydrophobic surface, creating a tiny bubble.

\NP Press Blank.

\NP Clean off surface, and repeate the previous step, but replace the elution buffer with each sample to be annalyzed.

\NP Load sample and press measure.

\NP Record the 260/280 and 260/230 ratios. They should be around 1.8-2.0 ideally. 


\section{References}

\NP APHA, AWWA. WEF. (2012) Standard Methods for examination of water and wastewater. 22nd American Public Health Association (Eds.). Washington. 1360 pp. (2014).

\end{document}
