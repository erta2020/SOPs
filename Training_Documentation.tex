\documentclass{article}\usepackage[]{graphicx}\usepackage[]{color}
%% maxwidth is the original width if it is less than linewidth
%% otherwise use linewidth (to make sure the graphics do not exceed the margin)
\makeatletter
\def\maxwidth{ %
  \ifdim\Gin@nat@width>\linewidth
    \linewidth
  \else
    \Gin@nat@width
  \fi
}
\makeatother

\definecolor{fgcolor}{rgb}{0.345, 0.345, 0.345}
\newcommand{\hlnum}[1]{\textcolor[rgb]{0.686,0.059,0.569}{#1}}%
\newcommand{\hlstr}[1]{\textcolor[rgb]{0.192,0.494,0.8}{#1}}%
\newcommand{\hlcom}[1]{\textcolor[rgb]{0.678,0.584,0.686}{\textit{#1}}}%
\newcommand{\hlopt}[1]{\textcolor[rgb]{0,0,0}{#1}}%
\newcommand{\hlstd}[1]{\textcolor[rgb]{0.345,0.345,0.345}{#1}}%
\newcommand{\hlkwa}[1]{\textcolor[rgb]{0.161,0.373,0.58}{\textbf{#1}}}%
\newcommand{\hlkwb}[1]{\textcolor[rgb]{0.69,0.353,0.396}{#1}}%
\newcommand{\hlkwc}[1]{\textcolor[rgb]{0.333,0.667,0.333}{#1}}%
\newcommand{\hlkwd}[1]{\textcolor[rgb]{0.737,0.353,0.396}{\textbf{#1}}}%
\let\hlipl\hlkwb

\usepackage{framed}
\makeatletter
\newenvironment{kframe}{%
 \def\at@end@of@kframe{}%
 \ifinner\ifhmode%
  \def\at@end@of@kframe{\end{minipage}}%
  \begin{minipage}{\columnwidth}%
 \fi\fi%
 \def\FrameCommand##1{\hskip\@totalleftmargin \hskip-\fboxsep
 \colorbox{shadecolor}{##1}\hskip-\fboxsep
     % There is no \\@totalrightmargin, so:
     \hskip-\linewidth \hskip-\@totalleftmargin \hskip\columnwidth}%
 \MakeFramed {\advance\hsize-\width
   \@totalleftmargin\z@ \linewidth\hsize
   \@setminipage}}%
 {\par\unskip\endMakeFramed%
 \at@end@of@kframe}
\makeatother

\definecolor{shadecolor}{rgb}{.97, .97, .97}
\definecolor{messagecolor}{rgb}{0, 0, 0}
\definecolor{warningcolor}{rgb}{1, 0, 1}
\definecolor{errorcolor}{rgb}{1, 0, 0}
\newenvironment{knitrout}{}{} % an empty environment to be redefined in TeX

\usepackage{alltt}
\usepackage{hyperref}

\hypersetup{
    colorlinks,
    citecolor=blue,
    filecolor=blue,
    linkcolor=blue,
    urlcolor=blue
}
\usepackage[margin=0.7in]{geometry}

\title{SOP Training Documentation}
\IfFileExists{upquote.sty}{\usepackage{upquote}}{}
\begin{document}


\maketitle

\begin{table}[h]
		\begin{tabular}{p{5cm}p{6cm}}
Researcher                  &  \framebox(200,15)[L]{}\\ 
Email Address               &  \framebox(200,15)[]{}\\
Expected Graduation Year    & \framebox(50,15)[]{} (NA for non-students)\\
		\end{tabular}
\end{table}

\section{Documentation Procedure}

\subsection{Background}

There are a numerous risks while doing laboratory and field based research. One challenge to reduce risks is the development of training records that can track training efforts. 

This form is used to track training in the Los Huertos lab. It is not a guareenttee that risks are no longer present, but that our lab has made a good effort to train all researchers and develop a laboratory where risks are better understood and strategies are in place to reduce these risks.

\section{Using this Document}

\subsection{Understanding SOPs}

For each procedure in the lab needed, researchers shall read and undersand the appropriate SOPs. 

After reading the researcher is responsible to communicate to the laboratory manager that they understand the intent, goals, and strategies of the SOP to reduce risks.

\subsection{Using Table \ref{tab:Documentation}}

After reading and disucssiong the SOP with the laboratory manager, sign and date that you have read the SOP. By signing this form, you have agreed to the following: 

\begin{quote} 
I agree that I have been adequately trained to use (designated equipment). I understand all safety hazards and operating procedures, and if anything goes wrong, I know what to do. Any injuries or broken equipment that happen while I use of the machine will be my fault alone.
\end{quote}

Initials: \framebox(50,15)[]{}

\subsection{Obtaining the Laboratory Manager's Signature}
Once you have completed this, contact the laboratory manager to verify the training and discussion of the SOP. This stage may require further training and is complete only when she/he signs and dates this form at the appropriate boxes. 

\subsection{Making the Form Available}

Once the form has been signed, place the form in the SOP Traning Binder in the Lab. 

\begin{table}[h!]
	\caption{SOP Traning Documentation}
	\label{tab:Documentation}
		\begin{tabular}{|c|c|p{4cm}|p{2cm}|p{4cm}|p{2cm}|}\hline
SOP No. & Version No. & Signature & Date  & Manager & Date \\
\hline
1  & & & & & \\ [1.2ex]\hline
2  & & & & & \\ [1.2ex]\hline
3  & & & & & \\ [1.2ex]\hline
4  & & & & & \\ [1.2ex]\hline
5\footnotemark  & & & & & \\ [1.2ex]\hline
6  & & & & & \\ [1.2ex]\hline
  & & & & & \\ [1.2ex]\hline
  & & & & & \\ [1.2ex]\hline
  & & & & & \\ [1.2ex]\hline
  & & & & & \\ [1.2ex]\hline
  & & & & & \\ [1.2ex]\hline
  & & & & & \\ [1.2ex]\hline
  & & & & & \\ [1.2ex]\hline
  & & & & & \\ [1.2ex]\hline
  & & & & & \\ [1.2ex]\hline
  & & & & & \\ [1.2ex]\hline
  & & & & & \\ [1.2ex]\hline
  & & & & & \\ [1.2ex]\hline
  & & & & & \\ [1.2ex]\hline
  & & & & & \\ [1.2ex]\hline
  & & & & & \\ [1.2ex]\hline
  & & & & & \\ [1.2ex]\hline
  & & & & & \\ [1.2ex]\hline
  & & & & & \\ [1.2ex]\hline
		\end{tabular}

\end{table}
\footnotetext{Required}

\end{document}
