%SOP Template 
% Version 02 Added revision date
% Version 03 Added TOC and acknowledgements
%           New SOP3_alpha.cls


\documentclass[12pt]{../SOP3_beta}

\usepackage[english]{babel}
\usepackage{blindtext}
\usepackage{lipsum}

\title{Trilogy Laboratory Fluorometer}
\date{02/07/2017}
\author{Marc Los Huertos}
\approved{TBD}
\ReviseDate{\today}
\SOPno{X}

\usepackage{Sweave}
\begin{document}
\Sconcordance{concordance:Trilogy_Laboratory_Fluorometer_v03.tex:Trilogy_Laboratory_Fluorometer_v03.Rnw:%
1 242 1 50 0}


\maketitle

\section{Scope and Application}

\NP The scope of this SOP is train researchers in the use of the Trilogy Laboratory Fluorometer, a compact, multifunctional laboratory instrument that can be used for making fluoresence, absorbance, or turbidity measurements using the appropriate snap-in Optical Module. 

\NP The applications of this SOP are for...

\section{Summary of Method}

\NP The Trilogy Laboratory Fluorometer is a compact, multifunctional laboratory insturment that can be used for making fluoresence, absorbance, and turbidity measurements using the appropriate snap-in Optical Module. A color touch screen with simple menus makes for an intuitive user interface. 

\tableofcontents

\newpage

\section{Acknowledgements}

\section{Modules}
\NP There are several different modules available to the Trilogy Laboratory Fluorometer:

\begin{itemize}
  \item 1. The Absorbance Module accepts interchangeable filter paddles so measurements can be made at different wavelenghts in order to identify or place a samplein a particular class of compounds. The standard filter paddle wavelengths/bandwidths are: 560/10; 600/10 and 750/10 nm.
  \item The Turbidity Module uses an Infrared (IR) LED with a wavelength of 850 nm as required for reference method: ISO 7027/DIN EN 27027, "Water Quality- Determination of Turbidity". Using Infrared allows Turbidity to be measured at wavelengths that are not normally absorbed by organic matter thereby reducing susceptibility to interference by sample color. 
\end{itemize}

\NP When properly calibrated, the Trilogy Fluorometer will read out the actual concentration of the solution. Optical Modules contain the necessary light source and filters for the relevant application. 

\section{Fluorometer Operation}
\subsection{Measuring Samples}
\NP Biases and interferences can come from...

\section{Health and Safety}

\NP Describe the risk...


\subsection{Safety and Personnnel Protective Equipment}


\section{Personnel \& Training Responsibilities}

\NP Researchers training is required before this the procedures in this method can be used... 

\NP Researchers using this SOP should be trained for the following SOPs:

\begin{itemize}
  \item SOP01 Laboratory Safety
  \item SOP02 Field Safety
\end{itemize}

\section{Required Materials and Apparati}

\NP Item 1 w/catalog number!

\NP Item 2

\section{Reagents and Standards}

\section{Estimated Time}

\NP This procedure requires XX minutes...

\section{Sample Collection, Preservation, and Storage}

\section{Procedure}

\NP Prepare \dots

\NP

\section{Data Analysis and Calculations}

\section{QC/QA Criteria}

\section{Trouble Shooting}

\section{References}

\NP APHA, AWWA. WEF. (2012) Standard Methods for examination of water and wastewater. 22nd American Public Health Association (Eds.). Washington. 1360 pp. (2014).

\end{document}
