\documentclass[12pt]{../SOP2}
\usepackage[english]{babel}
\usepackage{blindtext}
\usepackage{lipsum}

%\documentclass{article}

%\documentclass[12pt]{~/github/SOPs/SOP_Template/SOP}

\title{Electrical Power in the Field}
\date{8/11/2016}
\author{Reseacher Name}
\approved{Los Huertos}
\SOPno{X}

\usepackage{Sweave}
\begin{document}
\Sconcordance{concordance:ElectricPowerintheField.tex:ElectricPowerintheField.Rnw:%
1 17 1 1 0 78 1}


\maketitle

\section{Scope and Application}

\NP Electrical power can come from infastructure sources, such as outdoor or indoor outlets, portable generators, or solar power.

\NP This documents outlines some of the risks associated with electricity in the field

\NP We develop strategies to mitigate these risks.

\section{Health and Safety}

\NP Getting electricity in the field has risks, thus it's important to develop mitigation plans

\NP Always read and follow the manufacturer's operating instructions before running
generator

\NP Engines emit carbon monoxide. Never use a generator inside your home, garage, crawl space, or other enclosed areas. Fatal fumes can build up, that neither a fan
nor open doors and windows can provide enough fresh air.

\NP Only use the generator outdoors, away from open windows, vents, or doors.

\NP Use a battery-powered carbon monoxide detector in the area you're running a generator.

\NP Gasoline and its vapors are extremely flammable. Allow the generator engine to cool at least 2 minutes before refueling and always use fresh gasoline. 

\NP Maintain your generator according to the manufacturer’s maintenance schedule for peak performance and safety.

\NP Never operate the generator near combustible materials.

\NP If you have to use extension cords, be sure they are of the grounded type and are rated for the application. Coiled cords can get extremely hot; always uncoil cords and lay them in flat open locations.

\NP Never plug your generator directly into your home outlet. If you are connecting a generator into your home electrical system, have a qualified electrician install a Power Transfer Switch.

\NP Generators produce powerful voltage - Never operate under wet conditions. Take precautions to protect your generator from exposure to rain and snow. 

\section{Personnel \& Training Responsibilities}

\NP \lipsum[1]

Students using this SOP should be trained for the following SOPs:

\begin{itemize}
  \item SOP 03 Field Safety
\end{itemize}


\section{Required Materials}

\subsection{Item 1 w/catalog number!}
\subsection{Item 2}

\section{Estimated Time}

\NP This will take XX minutes...

\section{Procedure}

\NP Prepare \dots

\NP If you do not plan to use the generator in 30 days, use fuel stabilizer with gas and drain carborator. 

\section{References}

\NP APHA, AWWA. WEF. (2012) Standard Methods for examination of water and wastewater. 22nd American Public Health Association (Eds.). Washington. 1360 pp. (2014).

\end{document}
