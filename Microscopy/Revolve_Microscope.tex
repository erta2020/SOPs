\documentclass{article}

\title{Standard Operating Procedure: Microscope Methods Using the Echo labs Revolve Microscope}
\author{Dmaia Curry and Aparna Chintapalli}


\usepackage{Sweave}
\begin{document}
\Sconcordance{concordance:Revolve_Microscope.tex:Revolve_Microscope.Rnw:%
1 10 1 1 0 240 1}


\maketitle

\section{Introduction}

This SOP is to provide an introduction on how to use the EchoLabs Revolve Microscope, which includes a range of options -- some of which are complex.

Why revolve?

\subsection{Microscopy: Theory and Practice}

Text needed here

\subsubsection{Phase Contrast}

Phase contrast takes advantage of the fact that different structures have different refractive indices, and so bend light and delay its passage through the sample by different amounts. The retardation of the light results in some waves being 'out of phase' with others, and so to the human eye a microscope in phase contrast mode effectively darkens or brightens particular areas to reflect this change.

Phase contrast is used extensively in optical microscopy, in both biological and geological sciences. In biology, it is employed in viewing unstained biological samples with the human eye, making it possible to distinguish between structures that are of very similar transparency.

In geology, phase contrast is exploited in a different way to highlight differences between mineral crystals cut to a standardised thin section (usually 30 $\mu$m) and mounted under a light microscope. Crystalline materials are capable of exhibiting double refraction, in which light rays entering a crystal are split into two beams that may exhibit different refractive indices, depending on the angle at which they enter the crystal. The phase contrast between the two rays can be detected with the human eye using particular optical filters. As the exact nature of the double refraction varies for different crystal structures, phase contrast aids in the identification of minerals.

\subsubsection{Florescence}

A fluorescence microscope is an optical microscope that uses fluorescence and phosphorescence instead of, or in addition to, reflection and absorption to study properties of organic or inorganic substances. The "fluorescence microscope" refers to any microscope that uses fluorescence to generate an image, whether it is a more simple set up like an epifluorescence microscope, or a more complicated design such as a confocal microscope, which uses optical sectioning to get better resolution of the fluorescent image.

Fluorescence microscopy requires intense, near-monochromatic, illumination which some widespread light sources, like halogen lamps cannot provide. Four main types of light source are used, including xenon arc lamps or mercury-vapor lamps with an excitation filter, lasers, supercontinuum sources, and high-power LEDs. 

The Echo Lab Revolve has... ??

Fluorophores lose their ability to fluoresce as they are illuminated in a process called photobleaching. Photobleaching occurs as the fluorescent molecules accumulate chemical damage from the electrons excited during fluorescence. Photobleaching can severely limit the time over which a sample can be observed by fluorescent microscopy. Several techniques exist to reduce photobleaching such as the use of more robust fluorophores, by minimizing illumination, or by using photoprotective scavenger chemicals.

\subsection{Inverted versus Uprigt Mode}

More text here...

\subsection{Some Useful Links}

%\href{Useful video}{https://www.youtube.com/watch?v=IcubmIsxa7w}

\section{Procedure}

\subsection{Starting the Micrscope}

\begin{enumerate}
  \item To turn on the microscope, press the power button located on the top left corner of the iPad or press the home button on the right of the screen.
  \item After swiping the screen, the overhead microscope light should turn on and an image should appear on the screen.
  
\begin{itemize}
  \item If the microscope light is not on or if you do not see a microscope image, you must go to the home screen and click on the “ECHO LABS” app located at the bottom left of the screen.
  \item If the screens reads “Light Off”, click on BF located at the top of the screen.
\end{itemize}

  \item Once you have accessed the app, the overhead light should be illuminated. You may now load your slide into the slide tray.
\end{enumerate} 
  
\subsection{Using the Microscope--Bright Field} 

\begin{enumerate}
  \item To focus your image use the large knob to the right of the iPad. The knob has two parts. Use the outer knob to adjust the focus and the inner knob to fine tune the focus.
  \item To move the slide tray, there is a longer, downward facing knob above the focus knob. To move the tray left or right use the lower knob. To move the tray away or towards you, use the upper knob.
  \item To adjust the brightness, contrast and color balance of the image use the sliding controls on the bottom right of the screen. To adjust the intensity of the light, use the flat knob facing upward located on the bottom right of the machine
  
\end{enumerate}

\subsubsection{Recording Images in Bright Field Mode}

\begin{enumerate}
  \item To take pictures of the image on the screen, tap the camera shutter. It is the white circle located on the left of the screen. Make sure the anti-shake feature, the hand icon located below the camera shutter, is on. When turned on, the icon should be orange.
  \item To look at pictures taken, go to the gallery by pressing the square icon located at the top left of the screen. Press done to go back to the microscope. 
  \item If you are currently in BF (brightfield) mode and wish to switch to Fluorescence, tap the FL icon located at the top of the screen.  
\end{enumerate}


\subsection{Using the Microscope--Phase Contrast}

\subsubsection{Recording Images in Phase Contrast Mode}



\subsection{Using the Microscope--Florescence}



\subsubsection{Recording Images in Florescence Mode}

\end{document}
