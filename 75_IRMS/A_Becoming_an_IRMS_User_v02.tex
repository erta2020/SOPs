%SOP Template 
% Version 01 Created SOP



\documentclass[12pt]{../SOP3_beta}\usepackage[]{graphicx}\usepackage[]{color}
%% maxwidth is the original width if it is less than linewidth
%% otherwise use linewidth (to make sure the graphics do not exceed the margin)
\makeatletter
\def\maxwidth{ %
  \ifdim\Gin@nat@width>\linewidth
    \linewidth
  \else
    \Gin@nat@width
  \fi
}
\makeatother

\definecolor{fgcolor}{rgb}{0.345, 0.345, 0.345}
\newcommand{\hlnum}[1]{\textcolor[rgb]{0.686,0.059,0.569}{#1}}%
\newcommand{\hlstr}[1]{\textcolor[rgb]{0.192,0.494,0.8}{#1}}%
\newcommand{\hlcom}[1]{\textcolor[rgb]{0.678,0.584,0.686}{\textit{#1}}}%
\newcommand{\hlopt}[1]{\textcolor[rgb]{0,0,0}{#1}}%
\newcommand{\hlstd}[1]{\textcolor[rgb]{0.345,0.345,0.345}{#1}}%
\newcommand{\hlkwa}[1]{\textcolor[rgb]{0.161,0.373,0.58}{\textbf{#1}}}%
\newcommand{\hlkwb}[1]{\textcolor[rgb]{0.69,0.353,0.396}{#1}}%
\newcommand{\hlkwc}[1]{\textcolor[rgb]{0.333,0.667,0.333}{#1}}%
\newcommand{\hlkwd}[1]{\textcolor[rgb]{0.737,0.353,0.396}{\textbf{#1}}}%
\let\hlipl\hlkwb

\usepackage{framed}
\makeatletter
\newenvironment{kframe}{%
 \def\at@end@of@kframe{}%
 \ifinner\ifhmode%
  \def\at@end@of@kframe{\end{minipage}}%
  \begin{minipage}{\columnwidth}%
 \fi\fi%
 \def\FrameCommand##1{\hskip\@totalleftmargin \hskip-\fboxsep
 \colorbox{shadecolor}{##1}\hskip-\fboxsep
     % There is no \\@totalrightmargin, so:
     \hskip-\linewidth \hskip-\@totalleftmargin \hskip\columnwidth}%
 \MakeFramed {\advance\hsize-\width
   \@totalleftmargin\z@ \linewidth\hsize
   \@setminipage}}%
 {\par\unskip\endMakeFramed%
 \at@end@of@kframe}
\makeatother

\definecolor{shadecolor}{rgb}{.97, .97, .97}
\definecolor{messagecolor}{rgb}{0, 0, 0}
\definecolor{warningcolor}{rgb}{1, 0, 1}
\definecolor{errorcolor}{rgb}{1, 0, 0}
\newenvironment{knitrout}{}{} % an empty environment to be redefined in TeX

\usepackage{alltt}

\usepackage[english]{babel}

\title{Becoming a IRMS User}
\date{2/12/2018}
\author{Marc Los Huertos and Kyle McCarty}
\approved{TBD}
\ReviseDate{\today}
\SOPno{75A v0.1}
\IfFileExists{upquote.sty}{\usepackage{upquote}}{}
\begin{document}


\maketitle

\section{Scope and Application}

\NP The scope of this SOP defines who can use the IRMS and the training required to be a user and super-user.

\NP The applications of this SOP are for researchers to learn how to use the Oxtoby Isotope Lab IRMS. Using the IRMS requires skills and attention to detail and users must be qualified to use the instruments. The lab manager does not have the time or capacity to run samples for researchers, but can train users to run their samples. Completing this SOP is the first step toward becoming a user or super-user.

\section{Summary of Training}

\NP This SOP is used to train potential users how to prepare and run sample on the IRMS. Since the Oxtoby lab is managed by a 1/2 manager, it's important the users are able to run the instruments independently -- but they are expensive and very involved, so we need to ensure that users are qualified.

\NP Training will typically include background reading of relevant topics, observation of an already trained user, supervised experience running the instrumentation, and finally a start-to-finish experience that either the technician or specfic Super-Users will, with their discretion, determine a trainee a user.

\tableofcontents

\newpage

\section{Acknowledgements}

\section{Definitions}

\NP Super-User -- is student, staff, or faculty memember who is qualified to run and perform minor maintanence on the IRMS, including, but not necessarily limted to, gas replacement, reactor exchange, needle exchange, etc.  

\NP User -- is a student, staff, or faculty member who has qualified to prepare and run the IRMS without supervsion.

\NP Student Researcher -- is generally going to be a student who either does not feel confident in becoming a user or simply doesn't have the time to invest in it. Although, a student researcher can conduct sample weighing, data reduction, and sequence creation.

\NP Delta V IRMS -- The Oxtoby Lab's Isotope Ratio Mass Spectrometer, model Delta V manufactured by ThermoFisher Scientific. 

\NP Flash IRMS EA -- Also know as the ``Flash'' or ``Flash EA." It is an elemental analyzer and one of the three peripherals for the Delta V IRMS. It combusts a multitude of sample types to produce varying gases for the IRMS to analyze. 

\NP Gasbench II -- Is another periphepheral for the IRMS that samples (usually in conjunction with an autosampler), treats, and transports sample gases from sealed vials.

\NP TC/EA -- Thermal Conversion/Elemental Analyzer; similar to the Flash EA but instead uses pyrolysis and much higher temperatures (approximately 1450*C) to convert sample material into gases analyzed by the IRMS.

\NP ConfloIV -- A unique peripheral to the IRMS as it is the hub for all plumbed gas lines from each of the other peripherals, reference gases, and the IRMS itself. The Conflo also conducts the proper dilutions of sample and reference gases that is needed for them to fall within the working range of the IRMS.

\section{Laboratory Policies}

\NP In order to be able to run the IRMS and its associated peripherals, the person must have done the prerequisite training to become a User or Super-user. 

\NP Users are required to run their own samples and should not rely on anybody to run their samples for them. If a user cannot find the time to run their samples, or cannot find another user to run them, samples and sequences should not be ran.

\NP The Oxtoby Isotope Laboratory implements a ``payment in kind" approach in order to continue its operation. We do not require payment per sample, but, depending on how much the instrumentation is used, we will expect consumables and gases to be replenished by the users/super users themselves. If you happen to have certain consumables that you would like to use (instead of the lab's) feel free to do so. These items may include tin/silver capsules, cell wells, weighing paper, gloves, standards, reagents, etc.

\NP Lab Access

\section{Estimated Time}

\NP Estimated time to become a user requires approximately 6 hours of observation, training, and minimally supervised runs. This does not include the time required to read relevant reading material such as manuals and standard operating procedures.

\section{Health and Safety Risks}

\NP Pressurized, Reactive, and Poisenous Gases -Hydrogen (H$_2$), Oxygen (O$_2$),Carbon Monoxide (CO$_2$), Sulfur Dioxide (SO$_2$)

\NP Acid Handling - 100\% Phosphoric Acid

\NP Risk of Burns - Hot reactors

\NP Puncture and Cut Related Wounds - Needles

\section{Personnel \& Training Responsibilities}

\NP Users will be held to high professional standards and violation will forfeit your privilege to use the lab.

\NP Researcher training is required before time can be scheduled to use the IRMS and its peripherals.

\NP Researchers using this SOP should be trained for the following SOPs:

\begin{itemize}
  \item SOP01 Laboratory Safety
%  \item SOP02 Field Safety
\end{itemize}

\section{Required Materials and Apparati}

\subsection{Safety and Personnnel Protective Equipment (PPE)}

\NP Lab Coat

\NP Safety Glasses

\NP Gloves

\subsection{Other}

\NP Item 1 w/catalog number!

\NP Item 2

\section{Reagents and Standards}

\subsection{Reference/Standard Gases}

\subsubsection{Tank Farm 1 (West wall most northern)}

\NP Helium (He) - Tank pressure should be above 500psi and regulated at 50psi.

\NP Nitrogen (N$_2$)

\NP Carbon Dioxide (CO$_2$)

\NP Oxygen (O$_2$)

\subsubsection{Tank Farm 2 (West wall most southern)}

\NP Hydrogen (H)

\NP Carbon Monoxide (CO)

\NP Hydrogen and Helium (H and He)

\NP Hydrogen and Carbon Dioxide (H and CO$_2$)

\NP Sulfur Dioxide (SO$_2$)

\subsection{Reagents for Flash EA}

\NP Reaction Column packing ((Table~\ref{column})add partnumbers for reagents and reactor parts)

\begin{table}[h]
\label{column}
\caption{column}
\centering
\begin{tabular}{lccc} \hline
Analysis      & XX    & Copper  & \\ \hline\hline
CN            & Yes   & No      & Yes \\ \hline


\end{tabular}
\end{table}



%\section{Sample Collection, Preservation, and Storage}

\section{Procedure}

\NP Read general background of how isotope ratio ms works...30 min

\NP Observe other user(s) operate ...

\NP Read hardware SOPs and software SOPs?

\section{Background}

These instruments can range from tens to hundreds of thousands of dollars, and repairs on these instruments can not only be expensive, but they can also cause a backup in jobs. Since it is a fee for service laboratory, clients that submit their samples expect high quality data returned to them in a timely manner so that they may finish their projects. However, if instruments go down, those samples must be placed on hold until the laboratory receives any required parts or they are able to troubleshoot and fix the instruments. It is essential that the laboratory technician using the machines knows how to properly use it, and can troubleshoot when problems arise. When the instrument is new, the instrument users must not only attend extensive training specifically for use of the instrument, but they must also become familiar with the operations manual.

Go into a brief IRMS theory.

Explain the interaction of the peripherals and IRMS.

\section{Time Management - Sequence Preparation}

\NP Determine the number of samples will be analyzed, how many accompanying standards will be needed (depending on your data correction scheme), and blanks. Keep in mind the autosampler carousel has only 32 spots.

\NP Analysis Time - Varies depending on method and analysis type. 

\subsection{FLASH EA}

\begin{description}

\item[Carbon ($^{13}$C)] 
takes approximately 5 minutes plus an additional minute or so for peak centering and magnet switching.

\item[NC dual method] 
takes approximately 7 mintues plus and additional minute or so for peak centering and magnet switching.

\item[NCS triple analysis] 
takes approximately 10 minutes and 45 seconds plus an additional minute or so for peak centering and magnet switching.

\end{description}

\subsection{GASBENCH}

\begin{description}

\item[Carbonates]

\item[Dissolved Inorganic Carbon (DIC)]

\item[Breath Gas Analysis]

\item[CO$_2$ in Atmospheric Concentrtions]

\item[Water Equilibration ($^{18}$O/$^{16}$O)]

\item[Water Equilibration ($^2$H/$^1$H)]

\end{description}

\NP Perform instrument tests to verify instrument is functioning properly.

\section{QC/QA Criteria}

\NP Evaluate data reduction requirements, linearity, zero enrichment test

\section{Trouble Shooting}

\section{References}

\NP APHA, AWWA. WEF. (2012) Standard Methods for examination of water and wastewater. 22nd American Public Health Association (Eds.). Washington. 1360 pp. (2014).

\end{document}
