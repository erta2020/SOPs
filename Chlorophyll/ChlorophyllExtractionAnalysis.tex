\documentclass{article}
%This area before it says begin is like the header area (preamble)
\author{Isaac}
\title{SOP for In Vitro Determination of Chlorophyll \textit{a} Concentrations by Fluorescence}

\usepackage{Sweave}
\begin{document}
\Sconcordance{concordance:ChlorophyllExtractionAnalysis.tex:ChlorophyllExtractionAnalysis.Rnw:%
1 13 1 1 0 189 1}

\maketitle

\section{Scope and Application}
\begin{description}
  \item[1.1] This method provides a procedure for the fluorometric determination of chlorophyll \textit{a} and its magnesium-free derivative, pheophytin \textit{a} in marine and freshwater phytoplankton.
  \item[1.2] This method is modified from the US EPA Method 445.0 and APHA Standard Methods for the Examination of Water and Wastewater, 22\textsuperscript{nd} Edition. 
\end{description}

\section{Summary of Method}
\begin{description}
  \item[2.1] Chlorophyll-containing phytoplankton in a measured volume of sample water are concentrated by filtering at low vacuum (13 cm Hg) through a glass fiber filter (Whatman GF/F). The pigments are extracted from the phytoplankton in 90\% acetone and to ensure thorough extraction of chlorophyll \textit{a}, are allowed to steep for at least 2hrs. The fluorescence of the sample is measured at the excitation wavelength of 485 nm and emission wavelenghts 685 / 50 nm. Sample fluorescence is measured before and after acidification with 0.1M HCl to obtain a corrected chlorophyll \textit{a} concentration. 
\end{description}

\section{Definitions}
\begin{description}
\item[3.1] \textbf{Stock Standard Solution (SSS)} --
\item[3.2] \textbf{Laboratory Reagent Blank (LRB)} --
\item[3.3] \textbf{Field duplicates} --
\item[3.4] \textbf{Quality Control Sample (QCs)} --
\end{description}

\section{Interferences}
\section{Safety and Personnnel Protective Equipment}
\section{Related Documents}
\section{Materials and Apparatus}
\section{Reagents and Standards}
\section{Collecting \& Storing Samples}
\section{Procedure}
\section{Data Analysis and Calculations}
\section{QC/QA Criteria}



\end{document}
