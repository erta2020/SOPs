\documentclass{article}
%This area before it says begin is like the header area (preamble)
\author{Isaac}
\title{SOP for In Vitro Determination of Chlorophyll \textit{a} Concentrations by Fluorescence}
\usepackage{gensymb}
\usepackage{Sweave}
\begin{document}
\Sconcordance{concordance:ChlorophyllExtractionAnalysis.tex:ChlorophyllExtractionAnalysis.Rnw:%
1 13 1 1 0 189 1}

\maketitle

\section{Scope and Application}
\begin{description}
  \item[1.1] This method provides a procedure for the fluorometric determination of chlorophyll \textit{a} and its magnesium-free derivative, pheophytin \textit{a} in marine and freshwater phytoplankton.
  \item[1.2] This method is modified from the US EPA Method 445.0 and APHA Standard Methods for the Examination of Water and Wastewater, 22\textsuperscript{nd} Edition. 
\end{description}

\section{Summary of Method}
\begin{description}
  \item[2.1] Chlorophyll-containing phytoplankton in a measured volume of sample water are concentrated by filtering at low vacuum (13 cm Hg) through a glass fiber filter (Whatman GF/F). The pigments are extracted from the phytoplankton in 90\% acetone and to ensure thorough extraction of chlorophyll \textit{a}, are allowed to steep for at least 2hrs. The fluorescence of the sample is measured at the excitation wavelength of 485 nm and emission wavelenghts 685 / 50 nm. Sample fluorescence is measured before and after acidification with 0.1M HCl to obtain a corrected chlorophyll \textit{a} concentration. 
\end{description}

\section{Definitions}
\begin{description}
\item[3.1] \textbf{Stock Standard Solution (SSS)} -- a solution prepared in the laboratory using reference materials purchased from a reputable commercial source.
\item[3.2] \textbf{Laboratory Reagent Blank (LRB)} -- an aliquot of reagent water (Milli-Q/SuperQ) or other blank matrices that are treated the same as the sample including exposure to all glassware, equipment, solvents, reagents, internal standards and surrogates that are used with other samples. The LRB is used to determine if method analytes or other interferences are present in the laboratory environment, reagents or apparatus. 
\item[3.3]\textbf{Field duplicates}-- Two separate samples collected at the same time and placed under identical circumstances and treated exactly the same throughout the field and laboratory procedures. Provide a measure of the precision associated with sample collection, preservation, storage and laboratory processing. 
\item[3.4] \textbf{Quality Control Sample (QCs)}-- A solution of known concentration obtained from a source external to the laboratory to check laboratory performance.  
\end{description}

\section{Interferences}
\begin{description}
\item[4.1] Any substance extracted from the filter or acquired from laboratory contamination that fluoresces in the red region of the spectrum may interfere in the accurate measurement of both chlorophyll \textit{a} and pheophytin \textit{a}.
\item[4.2] Spectral interferences resulting from the fluorescence of the accessory pigment chlorophyll \textit{b}, and the chlorophyll \textit{a} degradation product pheophytin \textit{a}, can result in the overestimation of chlorophyll \textit{a} concentrations. However, highly selective optical filters used in this method minimize these interferences. 
\item[4.3] Quenching effects are observed in highly concentrated solutions or in the presence of high concentrations of other chlorophylls and carotenoids. Samples should be diluted. 
\item[4.4] Fluorescence is temperature dependent with higher sensitivity occurring at lower temperatures. Samples, standards, LRBS (section 3.2) and QCs (section 3.4) must be at the same temperature to prevent errors and maximize precision. Analysis of samples at ambient temperatures is required in this method. 
\item[4.5] All photosynthetic pigments are light and temperature sensitive. Work must be performed in subdued light and all standards, QC materials and filter samples must be stored in the dark at -20\degree C to prevent degradation.
\end{description}

\section{Safety and Personnnel Protective Equipment}
\begin{description}
\item[5.1] sfjsf
\end{description}
\section{Related Documents}
\section{Materials and Apparatus}
\section{Reagents and Standards}
\section{Collecting \& Storing Samples}
\section{Procedure}
\section{Data Analysis and Calculations}
\section{QC/QA Criteria}



\end{document}
