%SOP Template 
% Version 02 Added revision date
% Version 03 Added TOC and acknowledgements
%           New SOP3_alpha.cls
% Added text from Kathryn's SOP
\documentclass[12pt]{../SOP3}\usepackage[]{graphicx}\usepackage[]{color}
%% maxwidth is the original width if it is less than linewidth
%% otherwise use linewidth (to make sure the graphics do not exceed the margin)
\makeatletter
\def\maxwidth{ %
  \ifdim\Gin@nat@width>\linewidth
    \linewidth
  \else
    \Gin@nat@width
  \fi
}
\makeatother

\definecolor{fgcolor}{rgb}{0.345, 0.345, 0.345}
\newcommand{\hlnum}[1]{\textcolor[rgb]{0.686,0.059,0.569}{#1}}%
\newcommand{\hlstr}[1]{\textcolor[rgb]{0.192,0.494,0.8}{#1}}%
\newcommand{\hlcom}[1]{\textcolor[rgb]{0.678,0.584,0.686}{\textit{#1}}}%
\newcommand{\hlopt}[1]{\textcolor[rgb]{0,0,0}{#1}}%
\newcommand{\hlstd}[1]{\textcolor[rgb]{0.345,0.345,0.345}{#1}}%
\newcommand{\hlkwa}[1]{\textcolor[rgb]{0.161,0.373,0.58}{\textbf{#1}}}%
\newcommand{\hlkwb}[1]{\textcolor[rgb]{0.69,0.353,0.396}{#1}}%
\newcommand{\hlkwc}[1]{\textcolor[rgb]{0.333,0.667,0.333}{#1}}%
\newcommand{\hlkwd}[1]{\textcolor[rgb]{0.737,0.353,0.396}{\textbf{#1}}}%
\let\hlipl\hlkwb

\usepackage{framed}
\makeatletter
\newenvironment{kframe}{%
 \def\at@end@of@kframe{}%
 \ifinner\ifhmode%
  \def\at@end@of@kframe{\end{minipage}}%
  \begin{minipage}{\columnwidth}%
 \fi\fi%
 \def\FrameCommand##1{\hskip\@totalleftmargin \hskip-\fboxsep
 \colorbox{shadecolor}{##1}\hskip-\fboxsep
     % There is no \\@totalrightmargin, so:
     \hskip-\linewidth \hskip-\@totalleftmargin \hskip\columnwidth}%
 \MakeFramed {\advance\hsize-\width
   \@totalleftmargin\z@ \linewidth\hsize
   \@setminipage}}%
 {\par\unskip\endMakeFramed%
 \at@end@of@kframe}
\makeatother

\definecolor{shadecolor}{rgb}{.97, .97, .97}
\definecolor{messagecolor}{rgb}{0, 0, 0}
\definecolor{warningcolor}{rgb}{1, 0, 1}
\definecolor{errorcolor}{rgb}{1, 0, 0}
\newenvironment{knitrout}{}{} % an empty environment to be redefined in TeX

\usepackage{alltt}

\title{Foreign Soil Handling and Permit}
\date{2/02/2018}
\author{Kathryn Hargan}
\approved{Marc Los Huertos}
\ReviseDate{\today}
\SOPno{05 v0.2}
\IfFileExists{upquote.sty}{\usepackage{upquote}}{}
\begin{document}


\maketitle

\section{Scope and Application}

\NP The scope of this SOP is train researchers for the proper handling, analyses, and disposal of foreign soils. 

\NP The importation and movement of untreated soil is considered by APHIS to be an extremely high risk activity. For this reason, several levels of review and approval are required before a permit can be provided. This SOP has been written to fulfill our permit requirements.

\NP This standard operating procedure (SOP) outlines important practices and methodologies for securely and safely handling sediments and soils collected from other countries. 

\NP These practices are mandatory for all personnel working with such soils and sediments at the Environmental Analysis Teaching and Research Laboratory, Pomona College. 

\NP Additionally, any person working in this lab, but not with foreign soils will have reviewed this SOP and be aware of all standard laboratory operations with foreign soils, such that a safe and secure laboratory is maintained.  

\NP The applications of this SOP are for soils collected to be analyzed using the ICP-MS, EA-IRMS, and other parameters, such as texture and organic matter content. 

\section{Summary of Method}

\NP This SOP summarizes procedures to collect, store, process, and dispose of soils form foreign sources. 

\NP Sediments and soils are dried in clean facilities and added to plastic tubes that are sealed with a 1-cm layer of 2-ton epoxy. 

\NP Dried sediments and soils are analyzed for total nitrogen, sulfur and carbon and stable isotope ratios (C, N, and S). After all radiometric and isotopic analyses are complete, soils and their ash are sterilized in a laboratory oven at 250\degree C for 24 hours and discarded to a solid waste facility.

\tableofcontents

\newpage

\section{Acknowledgements}

\section{Definitions}

\NP Foreign Soils include soils and sediments collected outside the US and its territories. 
\NP APHIS -- The Animal and Plant Health Inspection Service is a multi-faceted Agency with a broad mission area that includes protecting and promoting U.S. agricultural health, regulating genetically engineered organisms, administering the Animal Welfare Act and carrying out wildlife damage management activities. These efforts support the overall mission of USDA, which is to protect and promote food, agriculture, natural resources and related issues.

\section{Biases and Interferences}

\NP Not Applicable

\section{Health and Safety}

\NP Soil is strictly controlled under APHIS quarantine regulations 7 CFR 330 because it can readily provide a pathway for the introduction of a variety of dangerous organisms into the United States.

\NP Importation of soil into the United States from foreign sources is prohibited, and movement within the continental U.S. is restricted unless authorized by APHIS under specific conditions, safeguards and controlled circumstances described in a permit and/or compliance agreement.


\subsection{Safety and Personnnel Protective Equipment}

\NP Soils shall be handled with care and researchers will wear protective clothing, gloves and goggles. 

\NP Foreign Soils shall only be stored in Room 133, processed in dedicated areas in Room 134.


\section{Personnel \& Training Responsibilities}

\NP Researchers using this SOP should be trained for the following SOPs:

\begin{itemize}
  \item SOP01 Laboratory Safety
\end{itemize}

\NP All personnel working at the Environmental Analysis Teaching and Research Laboratory must have taken Pomona College's lab safety course and training.

\NP Personnel protective equipment including a standard laboratory coat, nitrile gloves and lab safety goggles must always be worn when handling sediment and soil. 

\NP If any chemicals are to be utilized, work safety data sheets must be reviewed first. Students will obtain additional training and supervision from staff and faculty. 

\NP Training is required before any researcher uses foreign soils. Once approved, the research will have access to the foreign soil cabinet and refrigerator where soils can be analyzed for a range of parameters if they can be disposed of properly.  

\section{Required Materials and Apparati}

\NP Drying Oven (SGM 133)

\NP Log Book (Part No \# TBD).

\section{Reagents and Standards}

\NP Not applicable

\section{Estimated Time}

\NP Soils can only be present while an APHIS Foreign Soil Permit remains on record and within it's experation date.

\NP Collection of samples is expected to occur at several times throughout 2018 and 2019. 

\NP Estimated time from collection of samples until completion of sample analyses will be 2 years, until approximately 2020-2021. 

\NP Sample sterilization will occur throughout these 2 years and be finalized on remaining sediment and soils at the end of project. 

\NP In 2018, sample collection is expected to occur in March and May at the Kung Krabaen Bay Royal Development Study Centre in Chanthaburi province in eastern Thailand.

\section{Sample Collection, Preservation, and Storage}

\subsection{Importing Foreign Soils -- Shipping}

\NP All equipment will be thoroughly cleaned and disinfected of all soil residues at the collection site by soaking in isopropyl alcohol for one hour, or soaked and washed in hot water with detergent for 30 minutes. 

\NP Soil and sediment samples will be shipped in sturdy, leak-proof containers, and declared as ``Soil samples.'' 

\NP A shipping label is a tag that identifies an individual shipment. These labels direct the shipment to a designated port of entry or Plant Inspection Station (PIS) where shipments are inspected and cleared by Customs and Border Protection (CBP) and/or a United States Department of Agriculture (APHIS) inspector. The permittees permit will state in the conditions if a shipping label is required.

\NP Labels may be issued to the permittee for the importation of regulated articles. Such labels may contain information about the shipment's nature, origin, movement conditions, or other matters relevant to the permit and will indicate that the importation is authorized under the conditions specified in the permit.

\NP Regulated soil must be shipped in securely closed, watertight container(s) (primary container-test tube, vial etc.) which must be enclosed in a second, durable watertight container (Secondary container). 

\NP Each container must be able to contain the soil independently. 

\NP Imported shipments of 3lbs or less that will be treated at the(PIS) must have an original, Green and Yellow label (PPQ Form 508) attached to the exterior of each shipping package. 

\NP Shipments being routed directly to USDA approved facilities require Black \& White label (PPQ Form 550) Packages without labels on the exterior may be refused entry even if the labels are enclosed. 

\NP Each label is individually numbered with a distinctive barcode. The permittee must request the appropriate shipping labels at the issuance of permit. 

\NP Shipping labels and detailed instructions for using the labels will be emailed to the permittee as an attached PDF to the email address on record. 

\NP The PPQ 508 and 550 shipping labels are required for importation, but not for domestic movement within the mainland U.S. 

Note: PPQ 550 shipping labels are not needed for soil that originates in Hawaii, Puerto Rico, or the U.S. Virgin Island, i.e. they are not foreign countries.

\NP Additional labels can be obtained by using the My Shipments/Labels feature in ePermits or sending an email to \url{BlackWhiteGreenYellowlabelrequest@aphis.usda.gov}.

\NP A PPQ Form 508 shipping label requires two waybills; one international waybill to direct the shipment to the PIS, and a domestic waybill that will direct the shipment to the permittee from the PIS. 

\NP A domestic waybill should be placed inside the package. Shipments routed directly to a USDA approved facility require PPQ Form 550 shipping label and one waybill to direct the shipment to the permittee. 

\NP If the addresses don't match, the package may be refused entry by CBP. The information on the waybill will include the following: 

\begin{itemize}
\item Carrier account number or postage in order to forward the package to the final destination; 
\item Your name;
\item permit number; 
\item label number; 
\item delivery address; and 
\item telephone number of the permit holder for delivery following inspection.
\end{itemize}

\subsection{Hand Carry}

\NP Permitted articles may be hand carried into the United States only if certain requirements are met: 

\begin{itemize}
  \item Hand carrying must be requested in the application and authorized in the permit conditions;
  \item The permit conditions will list the specific steps that must be taken in order to hand-carry the articles into the U.S. 
  \item The applicant may request that someone else be allowed to hand carry the organisms but the person must be identified on the permit.
  \item The authorized hand-carrier must follow the permit conditions pertaining to the hand-carried shipment.
\end{itemize}

\subsection{Final Destination Processes}

\NP Soils must be directly transported to the Biogeochemistry Lab to be stored in the Foreign Soil Cabinet (SGM 133). Under no circumstances are soils to be opened or otherwise processed until reaching Pomona's foreign soil cabinet. 

\NP When received the containers will be opened and examined inside a closed room area. Lab personnel will take all precautions to prevent the escape of any potential pest which may be the soil. 

\NP Reshipment of soil samples to other labs will not occur unless sterilized. A detailed request to reship these soils will be submitted to the enforcing agricultural inspectors. 

\subsection{Storage and Logging Soils}

\NP Upon arrival at Pomona College, all samples are logged into the \texttt{Soils Log}.

\NP Each sample must be logged with the following information: sample provenience, reasons for collection, mass, and types of analysis performed.

\NP Soil will be kept in quarantine conditions and in closed containers until sterilized inn SGM 133. The soil will be kept in a designated storage area labeled ``Quarantine Soil`` when not in use.

\section{Procedure}

\subsection{Sample labeling and identification}

\NP Each sample receives a separate number, consisting of the year and a consecutive number for each sample (e.g., 17-34, indicating the 34th sample logged in 2017).

\NP For each sample, record in log book: Date of Arrival, Origin, and Amount (in grams). Also record name/initials of who collected the sample, and where stored (SGM 133, Pomona College, fridge or freezer number if appropriate). 

\NP Samples remain sealed in original whirlpac bags within Ziploc bag until processed, and bags are stored securely on shelf trays, within closed container, or within cabinets. Samples should not be left on tables or elsewhere when not in use.

\NP Sterilized soil samples (see below) must be labeled with green tags stating, ``Quarantine Soil -Sterilized.'' Attach labels to trays or boxes where samples are stored, making sure tag is prominently displayed.

\NP All stored, unsterilized soils must be labeled with red tags stating ``Quarantine Soil - Unsterilized.''

\NP Any individual soil samples that are out of their storage area and/or being processed must also be clearly labeled to indicate whether or not they have been sterilized.


\subsection{Treatment and Disposal of Soils}

\NP Soil samples must be sterilized by dry heat treatment at 250\degree C for 24 hrs, or by one of the other approved methods (see additional documentation in \texttt{Soils Log}).

\NP All residues must be properly sterilized in the same manner.

\NP Once emptied, all containers (shipping boxes, plastic bags) must be sterilized/disposed of properly by burning, or if containers are to be re-used they must be decontaminated by an approved heat treatment method or disinfection with isopropyl alcohol.

\NP Whenever water is utilized in processing a sample, including initial rinse water of contaminated equipment, the contaminated water shall be disposed of by one of the following methods: 

\begin{itemize}
  \item Boiling the water for 1 minute; 
  \item Treating with a strong detergent for 30 minutes; or 
  \item filtering through a 100-mesh screen or suitable paper filter. The residues left in the filter should be burned or sterilized.
\end{itemize}


\section{Soil Survellience and Logging}

\NP In addition to data logged upon arrival (see above), log entries for each sample must include method of treatment, mass treated, and date of disposal (after sample has been processed).

\NP The mass of the treated and disposed soils must be within 10\% of the imported soil.


\section{References}

\NP APHA, AWWA. WEF. (2012) Standard Methods for examination of water and wastewater. 22nd American Public Health Association (Eds.). Washington. 1360 pp. (2014).

\end{document}
