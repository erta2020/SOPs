\documentclass[12pt]{../SOP2}
\usepackage[english]{babel}
\usepackage{blindtext}
\usepackage{lipsum}

%\documentclass{article}

%\documentclass[12pt]{~/github/SOPs/SOP_Template/SOP}

\title{Analysis of Soil Pb}
\date{3/11/2017}
\author{Marc Los Huertos \& Katheryn Kornegay}
\approved{Los Huertos}
\ReviseDate{\today}
\SOPno{35}

\usepackage{Sweave}
\begin{document}
\Sconcordance{concordance:Soil_Pb.tex:Soil_Pb.Rnw:%
1 16 1 1 0 135 1}


\maketitle

\section{Scope and Application}

\NP This SOP describes the extraction process of metals from soil. With each method of metal extraction, results can vary depending on properties of the soil and speciation of metals. This method is \ldots

\NP The SOP can be applied to \ldots

\tableofcontents

\newpage

\section{Health and Safety}

\NP The method relies on extracting soils in hot, i.e. boiling concentrated acid. Thus, maintaining a safe laboratory environment is key to the success of this SOP. 


\section{Personnel \& Training Responsibilities}

\NP To be qualified on for this SOP students must be appropriately trained.

The technician must be approved for the following SOPs:

\begin{itemize}
  \item SOP 01 Laboratory Safety
  \item SOP 02 Handling of Hazardous Material
\end{itemize}


\section{Preparation}

\subsection{Supplies and Equipment}

\begin{itemize}
  \item Mason jar with screw lids -( Not using this…too messy?) OR
  \item Mesh sieves -(Sizes: No.10; Locations: entry room of the lab; Cabinet: Ziploc Bags \& Notebooks) total amount: 4 
  \item balance -(Mettler Toledo Precision Balance, Model MS1602TS, Made in Switzerland,on the side table)
  \item 25433-008 VWR 60 mL Disposable Aluminum Crinkle Dishes with Tabs 332.60 dollars \href{https://us.vwr.com/store/catalog/product.jsp?product_id=4622693}{VWR Disposable Aluminum Crinkle Dishes with Tabs}
  \item Fisherbrand 25mm syringe filter 0.45 um, Cat no 09-719D \href{https://www.emdmillipore.com/US/en/product/MF-Millipore%E2%84%A2-Membrane-Filters,MM_NF-C152}{MF-Millipore Membrane Filters}
  \item Centrifuge tubes (whatever used at Keck)
  \item Tongs -(Cabinet 9: Mattle Tins, Plastic Bottle)
  \item Hot plate -(Thermo Scientific Cimarec S88857104 Stirrer, 7x7" Aluminum; 120 VAC, Assembled in China, inside the Fumehoods)
  \item Funnel -( 64mm, 10500, 12PK, Made in Mexico, The White table with water sinks; the top drawer by the wall)
  \item Glass wool -(ran out); CAS Number 65997-17-3 \href{http://www.sigmaaldrich.com/catalog/product/sial/18421?lang=en&region=US}{Sigma Aldrich Catalog}
  \item 100ml volumetric flask- (Back cabinet: volumetric flasks \& graduated cylinders; there are 17)
  \item Stop watches 
  \item Gloves
  \item 250ml Erlenmeyer Flask- (Back cabinet: erlenmeyer flasks \& beakers; there are 21)
\end{itemize}

\subsection{Reagents}

\begin{itemize}
  \item Nitric acid
\end{itemize}

\section{Estimated Time}

\NP This will take XX minutes...

\section{Procedure}
\NP Part II: Determination of Lead levels using Atomic absorption spectroscopy

***You must wear safety goggles or eyeglasses***
Plastic Gloves are available to protect your hands from acid. Wash off any acid, even on the gloves.
\begin{itemize}
  \item A. Preparation of the soil samples:
  \begin{itemize}
    \item 1. From the Mason jar with mesh on top, sieve approximately 10.0 g soil on to the plastic weighing dish.
    \item 2. Transfer all the weighed soil to a 250-mL ehrlemeyer (conicle-y shaped flask). Label the flask with your initials on label tape. 
  \end{itemize}
  \item B. Extracting the Lead:
  \begin{itemize}
    \item 3. In the fume hood - Add 20 mL of 1M nitric acid CAREFULLY (slowly) to the soil.
    \item 4. Heat the mixture near boiling on the hot plate (medium setting) until the fumes are almost colorless. Heat for 5 -10 additional min. Do not let the mixture go dry - add a few mLs of water if needed. At end, with tongs, remove flask from plate, and cool 5-10 min. Then take back to bench.
    \item 5. CAREFULLY add 75 mL of deionized water to the flask and swirl GENTLY. Let stand 3-5 min so heavier particles can  settle out.
    \item 6. Take a funnel and put a small wad of glass wool in the funnel and place the funnel in a 100-mL volumetric flask.    Label this flask with your initials on tape.
    \item 7. Carefully transfer the acid/soil solution from the conical flask to the volumetric flask. You will not remove all the soil particles, but most of the larger ones will be filtered out.
    \item 8. Fill the volumetric flask to the circular mark on the neck with deionized water. Mix well by putting your thumb over the stopper and inverting the flask several times. Let stand several minutes to allow soil to settle down. Slowly pour 15 mL of the water at the top of the volumetric flask into the centrifuge tube to take over the KSD main building.
    \begin{itemize}
      \item SO get centrifuge tubes and then move them to Keck?
  How are we going to store the tubes if necessary?
  How are we moving the tubes to Keck?
  Could it be better to do the whole thing at Keck?
    \end{itemize}  
  \end{itemize}
  \item C. Calibrating Solutions:
  \begin{itemize}
    \item 9. Blank Solutions: \href{https://en.wikipedia.org/wiki/Blank_(solution)}{Blank Solutions Explanation}
    \item 10. Standard Solutions: \href{https://en.wikipedia.org/wiki/Standard_solution}{Standard Solutions Explanation}
    \item 11. Spiked Solutions: \href{}{Spiked Solutions Explanation}
  \end{itemize}  
  \item D. Determining the lead content:
  \begin{itemize}
    \item 12. Centrifuge a 15-mL sample until solution is clear. Usually 3 to 4 min. Use a syringe and Millipore filter to push (gently) about 10 mL of solution through the filter into the small labeled sample vial.
    \item 13. Measure the concentration of the solution on the Atomic Adsorption Spectrometer.  Multiply the reading for your sample by 10 to calculate the lead concentration in your soil sample.  The final result is in “ppm”, which means milligrams (0.0010 g) of lead per kilogram of soil.
  \end{itemize}
  \item E. Dispose of sample and solutions as directed by instructors. Clean up your bench space.
\end{itemize}


\section{References}

\NP APHA, AWWA. WEF. (2012) Standard Methods for examination of water and wastewater. 22nd American Public Health Association (Eds.). Washington. 1360 pp. (2014).

\end{document}
