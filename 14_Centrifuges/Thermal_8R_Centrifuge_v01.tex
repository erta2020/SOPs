%SOP Template 
% Version 02 Added revision date
% Version 03 Added TOC and acknowledgements
%           New SOP3_alpha.cls


\documentclass[12pt]{../SOP3_beta}

\usepackage[english]{babel}
\usepackage{blindtext}
\usepackage{lipsum}

\title{Thermal 8R Centrifuge SOP}
\date{X/XX/XXXX}
\author{Micah Maglasang}
\approved{TBD}
\ReviseDate{\today}
\SOPno{X}

\usepackage{Sweave}
\begin{document}
\Sconcordance{concordance:Thermal_8R_Centrifuge_v01.tex:Thermal_8R_Centrifuge_v01.Rnw:%
1 19 1 1 0 168 1}


\maketitle

\section{Scope and Application}

\NP The scope of this SOP is train researchers how to properly handle the Thermal 8R Centrifuge

\NP The centrifuge is a laboratory product used to separate substance mixtures of different densities.  

\section{Summary of Method}

\NP This SOP does this...

\tableofcontents

\newpage

\section{Acknowledgements}

\section{Definitions}

\NP Term1: is...

\section{Biases and Interferences}

\NP Biases and interferences can come from...

\section{Health and Safety}

\NP Observe the safety instructions. Not following these instructions can cause damage

\NP The centrifuge is to be used for its intended use only. Improper use can cause damages, contamination, and injuries with fatal consequences

\subsection*{Set Up Conditions}

\NP Plug the centrifuge only into sockets which have been properly grounded

\NP Turn off the centrifuge at the main switch. The main plug must be freely accessible at all times.

\NP Press the STOP key to shut down the centrifuge.

\NP Pull out the power supply plug or disconnect the power supply in an emergency.

\NP As safety zone maintain a clear radius of at least 30 cm around the centrifuge.

\NP Do not place any dangerous substances within this security zone. 

\NP Set up in a well-ventilated environment, on a horizontally levelled and rigid surface with adequate load-bearing capacity.

\subsection {Preparation}

\NP Make sure to follow the "Laboratory Biosafety Manual" of the World Health Organization (WHO) and the regulations in your country.

\NP Do not make any changes to the mechanical components of the rotor.

\NP Do not touch the electronic components of the centrifuge nor alter any electronic or mechanical components.

\NP Use only with rotors which have been properly installed.

\NP Do not use rotors, buckets or accessories which show any signs of removed protective coating, corrosion or cracks. 

\NP Use only with rotors which have been loaded properly.

\NP Never overload the rotor.

\NP Always balance the samples.

\NP Make sure the rotor is locked properly into place before operating the centrifuge.

\NP Implement measures which ensure that no one can approach the centrifuge for longer than absolutely necessary while it is running.

\subsection {Hazardous Substances}

\NP Especially when working with corrosive samples (salt solutions, acids, bases), the accessory parts and vessel have to be cleaned thoroughly.

\NP Do not centrifuge explosive or flammable materials or substances.

\NP The centrifuge is neither inert nor protected against explosion. Never use the centrifuge in an explosion-prone environment.

\NP Do not centrifuge toxic or radioactive materials or any pathogenic micro-organisms without suitable safety precautions.

\NP If toxins or pathogenic substances have contaminated the centrifuge or its parts, appropriate disinfection measures have to be taken.

\NP Extreme care should be taken with highly corrosive substances which can cause damage and impair the mechanical stability of the rotor. These should only be centrifuged in fully sealed tubes.

\NP If a hazardous situation occues, turn off the power supply to the centrifuge and leave the area immediately.

\subsection {Operating}

\NP Never use the centrifuge ifn parts of its cover panels are damaged or missing.

\NP Never start the centrifuge when the centrifuge door is open

\NP Do not move the centrifuge while it is running.

\NP Do not lean on the centrifuge.

\NP Do not place anything on top of the centrifuge during a run.

\NP Never open the centrifuge door until the rotor has come to a complete stop and this has been confirmed in the display.

\NP The emergency door release may be used in emergencies only to recover samples from the centrifuge, e.g. during a power failure.

\NP Do not open the centrifuge while it is running

\NP In any case of sever mechanical failure, such as rotor or bucket crash, the centrifuge is not aerosol-tight.

\NP In any case of rotor faulure the centrifuge can be damaged. Leave the room and contact customer service.

\section{Personnel \& Training Responsibilities}

\NP Researchers training is required before this the procedures in this method can be used... 

\NP Researchers using this SOP should be trained for the following SOPs:

\begin{itemize}
  \item SOP01 Laboratory Safety
  \item SOP02 Field Safety
\end{itemize}

\section{Required Materials and Apparati}

\NP Thermo Scientific(TM) Sorvall(TM) 8/8R centrifuge (catalog number?)

\NP Power Supply Cable

\NP Instruction Manual

\NP CD

\section{Reagents and Standards}

\section{Estimated Time}

\NP This procedure requires XX minutes...

\section{Sample Collection, Preservation, and Storage}

\section{Procedure}

\subsection {Transportation and Set Up}

\subsection {Location}

\NP The centrifuge is only to be operated indoors.

\NP A safety zone of at least 30 cm must be maintained around the centrifuge.

\NP The supporting structure must be stable and free of resonance.


\NP The supporting structure must be suitable for horizontal setup of the centrifuge.

\NP The centrifuge is not to be exposed to heat and strong sunlight.

\NP The set-up location must be well-ventilated at all times.


\section{Data Analysis and Calculations}

\section{QC/QA Criteria}

\section{References}

\NP APHA, AWWA. WEF. (2012) Standard Methods for examination of water and wastewater. 22nd American Public Health Association (Eds.). Washington. 1360 pp. (2014).

\end{document}
