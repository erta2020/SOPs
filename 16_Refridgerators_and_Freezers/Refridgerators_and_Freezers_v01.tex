%SOP Template 
% Version 02 Added revision date
% Version 03 Added TOC and acknowledgements
%           New SOP3_alpha.cls


\documentclass[12pt]{../SOP3_alpha}

\usepackage[english]{babel}
\usepackage{blindtext}
\usepackage{lipsum}

\title{Refridgerators and Freezers}
\date{8/15/2016}
\author{Marc Los Huertos}
\approved{TBD}
\ReviseDate{\today}
\SOPno{16 v.01}

\usepackage{Sweave}
\begin{document}
\Sconcordance{concordance:Refridgerators_and_Freezers_v01.tex:Refridgerators_and_Freezers_v01.Rnw:%
1 19 1 1 0 114 1}


\maketitle

\section{Scope and Application}

\NP The scope of this SOP is for the laboratory refigerator, freezer (-30) and high XX freezer (-80).

\NP The applications of this SOP is to storage and what to put in which freezer...

\section{Summary of Method}

\NP This SOP does this...

\tableofcontents

\newpage

\section{Acknowledgements}

\section{Definitions}

\NP Term1: is...

\section{Interferences}

\NP Biases and interferences can come from...

\section{Health and Safety}

\NP The refrigerator and freezers in the laboratory are not approved to store flammable material. In fact, these instruments can generate signficant static electricity that can cause an explosion when there are flammable fumes present. 

\NP No food shall be stored in the laboratory refrigerator or freezers.


\subsection*{Safety and Personnnel Protective Equipment}


\section{Personnel \& Training Responsibilities}

\NP Researchers training is required before this the procedures in this method can be used... 

\NP Researchers using this SOP should be trained for the following SOPs:

\begin{itemize}
  \item SOP01 Laboratory Safety
  \item SOP02 Field Safety
\end{itemize}

\section{Required Materials and Apparati}

\NP Item 1 w/catalog number!

\NP Item 2

\section{Reagents and Standards}

\section{Estimated Time}

\NP This procedure requires XX minutes...

\section{Sample Collection, Preservation, and Storage}

\section{Procedure}

\NP Prepare \dots
\subsection{4\degree Refrigerator}

\NP 

\subsection{-30\degree Freezer}

\NP 

\subsection{-80\degree Freezer}

\NP Only open the Freezer when you are ready to put samples in or remove samples. 

\NP Wear PPE for extreme cold that includes insulated mittens and a laboratory coat.

\NP When you store samples in the freezer, you MUST document the locations that you store them. 

\NP Record the following information in the \href{https://docs.google.com/spreadsheets/d/1ohQFRFMBHCu2Wm8H58UH67SJ_uvwSsbRFGZVjZr9Bjc/edit?usp=sharing}{Google doc} and in your laboratory book.

\begin{itemize*}
  \item Shelf (Each shelf is contained behind an inner door)
  \item Divider
  \item Drawer
  \item Box
  \item Box location
  \item Project
  \item Date (Collected from Field)
  \item Sample ID
  \item Sample Description, including nanodrop results.
\end{itemize*}


\section{Data Analysis and Calculations}

\section{QC/QA Criteria}

\section{References}

\NP APHA, AWWA. WEF. (2012) Standard Methods for examination of water and wastewater. 22nd American Public Health Association (Eds.). Washington. 1360 pp. (2014).

\end{document}
