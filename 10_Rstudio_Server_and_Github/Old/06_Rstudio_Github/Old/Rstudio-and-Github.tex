\documentclass{article}
\usepackage{hyperref}

\author{Marc, Aparna, and Isaac}
\title{Setting up Rstudio Projects and Github}

\usepackage{Sweave}
\begin{document}
\Sconcordance{concordance:Rstudio-and-Github.tex:Rstudio-and-Github.Rnw:%
1 6 1 1 0 92 1}

\maketitle

\section{Introduction}

R is a powerful, open source program but combined with RStudio and Github the program becomes an archetype of a program that enables 1) collaboration, 2) transparency, and 3) accessibility. 

However, becoming facile in using these program is like learning how to walk. We need to approach this process in descrete steps. Some will require weeks of mistakes to learn, others will quickly learn to run with the programs. How quickly you can feel comfortable with these programs will depend on many factors, but will be greatly improved by the time you invest!

\subsection{Rational}

\begin{description}
  \item[Almost Universal Compatibility] No downloading programs on personal computers.
  \item[Accessibility] Accessible from any computer via webserver
  \item[Collaboration]
  \item[Version Control]
\end{description}

\section{Rstudio}

Although R by itself is an amazing example of crowd sourcing, where a wide range of staticians and programmers have created a free programming environments with a robust range of statistical packages, the RStudio interface provides a user with the tools to track and publish their analysis process in an effecient and transparent way. 

\subsection{Local Install versus Server}

R and RStudio can be installed on a local machine from the CRAN download micro sites. However, you also have access to the RStudio Server that has been installed on the Pomona College mainframe, where you can access it via a web browser. Wow, this is conveient!

\section{Create Version Control for Rstudio as a Collaborator}

The final component of R and RStudio is its capacity to create projects (RStudio's terminology) and repositories (Github's terminalogy) that can be shared among collaborators. 

In particular, the collaboration allows for contributions to be tracked via version control tools. There are a number of ways that we can access these tools, but we'll try to limit the methods so keep the process relatively "simple".

\begin{description}
  \item[Create Github Account] Go to \href{http:\\github.com}{Github.com} and create an account. I suggest you use your Claremont e-mail adddress because this can come in handy later, should you want to create private repositories, which are free for college students! Your email address can be used to prove your student status.
  
  \item[Accept Collaborator Invitation] Click on the email link to accept the invitation from the collaborator. Once you have done this, you have access to all the files created in the repository. 
  
  \item[Read the Readme file]

  \item[Make SSH Key on Rstudio Sever] As a collaborator, you need to create a SSH Key that is used to create secure connections between RStudio and Github. To obtain an SSH key, go to RStudio/Tools/Global Options and select "Create an SSH key..."
  
  Profile
  \item[Add SSH Key to Github] Copy SSH key from Rstudio, go to GitHub settings. Under SSH and GPG key, click on new SSH key, type in name of key (e.g. Rstudio server) and paste in the RSA key into the window below. Hit green "add SSH key" button...
  \item[Clone] shared project and paste into New Project on R Studio Server
\end{description}

\section{Creating a New Repository}

\subsection{Creating a New Project}

\section{Creating a Project as a Collaborator}


\section{Working on Projects}

\subsection{Rstudio Panels}

\begin{description}
  \item[File Structure and Rnw/Tex files]
  \item[Console and Compile PDF windows]
  \item[Git Panel]
\end{description}


\subsection{Best Practices}

\begin{description}
  \item[Pull] When you open RStudio, the first thing you should do is "Pull" from the repository to ensure your files are up-to-date. When you "Pull", you will get one of three results:
  
  \begin{description}
  \item[Already up-to-date.] This means that your files have not been updated on the Github site since you last pulled the files. If you suspect someone has worked on the files, but are not getting those changes, it means that your collaborator has failed to "Push" these changes onto the repsository. If this is the case, you might need to go to the troubleshooting question to address this problem.
  \item[Successful Updates] If your files are successfully updated from the repository, you will see:
  
summary of updates

master 6038765 Started to explain set up procedures
 
 5 files changed, 434 insertions(+), 8 deletions(-)
 
 create mode 100644 06\_Rstudio\_Github/Rstudio-and-Github-concordance.tex
 
 create mode 100644 06\_Rstudio\_Github/Rstudio-and-Github.log
 
 create mode 100644 06\_Rstudio\_Github/Rstudio-and-Github.pdf
 
 create mode 100644 06\_Rstudio\_Github/Rstudio-and-Github.tex
 
 \item[Unsuccesful "Pull"]
 
 
\end{description}

  \item[Commit]
  \item[Push] As you might guess, when you "Push" you can also have several outcomes:
\begin{description}
  \item[Everything up-to-date] is certainly simple.   \item[Successful "Push"]
  
    "To git@github.com:marclos/SOPs.git
  
   e2efe6f..0b18250  master -> master"
   \item[Unsuccessful "Push"]
\end{description}
  
  
  

   
\end{description}

\section{Troubleshooting}
At first when you're a newcomer to working on projects within Rstudio connected to a GitHub repository, you may forget the "best practices" of pulling, committing and pushing described above. You might have already gone through all the steps involved in setting up your GitHub account, linking your workspace in Rstudio with the correct project, and begun work on a specific file, but if you forget to update your workspace each time you come back to modify a file (especially in the case of coming back the next day to a file in Rstudio and forgetting to pull changes other collaborators may have made), you will run into problems committing, and pushing your changes. In this most common case, it's not likely that you'll notice any problems until you try to commit your changes. Thus, we'll begin this section on trouble shooting problems with Github and Rstudio at the committ level.

\subsection{"Commit" problems}
In the scenario described above, you will get an error code when trying to committ your new changes to a file. 

\subsection{"Pull" is rejected}

Merging changes... how??
\subsection{"Push" fails}

Below are several potential remedies:

\begin{description}
  \item[Deleting a Project in R Studio]If you are willing to sacrifice the changes you made or have mailed them to a collaborator to deal with you can delete the entire project in Rstudio. To accomplish this delete all files in directory, clear workspace and console, don't save, then go to session tab: Terminate R, and hopefully that will do the trick. Then commence with starting a new project. 

\end{description}

\end{document}
