\documentclass[12pt]{article}\usepackage[]{graphicx}\usepackage[]{color}
%% maxwidth is the original width if it is less than linewidth
%% otherwise use linewidth (to make sure the graphics do not exceed the margin)
\makeatletter
\def\maxwidth{ %
  \ifdim\Gin@nat@width>\linewidth
    \linewidth
  \else
    \Gin@nat@width
  \fi
}
\makeatother

\definecolor{fgcolor}{rgb}{0.345, 0.345, 0.345}
\newcommand{\hlnum}[1]{\textcolor[rgb]{0.686,0.059,0.569}{#1}}%
\newcommand{\hlstr}[1]{\textcolor[rgb]{0.192,0.494,0.8}{#1}}%
\newcommand{\hlcom}[1]{\textcolor[rgb]{0.678,0.584,0.686}{\textit{#1}}}%
\newcommand{\hlopt}[1]{\textcolor[rgb]{0,0,0}{#1}}%
\newcommand{\hlstd}[1]{\textcolor[rgb]{0.345,0.345,0.345}{#1}}%
\newcommand{\hlkwa}[1]{\textcolor[rgb]{0.161,0.373,0.58}{\textbf{#1}}}%
\newcommand{\hlkwb}[1]{\textcolor[rgb]{0.69,0.353,0.396}{#1}}%
\newcommand{\hlkwc}[1]{\textcolor[rgb]{0.333,0.667,0.333}{#1}}%
\newcommand{\hlkwd}[1]{\textcolor[rgb]{0.737,0.353,0.396}{\textbf{#1}}}%
\let\hlipl\hlkwb

\usepackage{framed}
\makeatletter
\newenvironment{kframe}{%
 \def\at@end@of@kframe{}%
 \ifinner\ifhmode%
  \def\at@end@of@kframe{\end{minipage}}%
  \begin{minipage}{\columnwidth}%
 \fi\fi%
 \def\FrameCommand##1{\hskip\@totalleftmargin \hskip-\fboxsep
 \colorbox{shadecolor}{##1}\hskip-\fboxsep
     % There is no \\@totalrightmargin, so:
     \hskip-\linewidth \hskip-\@totalleftmargin \hskip\columnwidth}%
 \MakeFramed {\advance\hsize-\width
   \@totalleftmargin\z@ \linewidth\hsize
   \@setminipage}}%
 {\par\unskip\endMakeFramed%
 \at@end@of@kframe}
\makeatother

\definecolor{shadecolor}{rgb}{.97, .97, .97}
\definecolor{messagecolor}{rgb}{0, 0, 0}
\definecolor{warningcolor}{rgb}{1, 0, 1}
\definecolor{errorcolor}{rgb}{1, 0, 0}
\newenvironment{knitrout}{}{} % an empty environment to be redefined in TeX

\usepackage{alltt}

\title{Quiz}
\author{EA31}
\IfFileExists{upquote.sty}{\usepackage{upquote}}{}
\begin{document}

\maketitle

\begin{enumerate}

\item The default prompt symbol is 

A. $>$\\
B. $->$ \\
C. $\&$ \\
D. $>>$ 

\item R relies on functions with specific syntax. What symbols are required for a function to work?

A. Semicolon \\
B. Periods \\ 
C. Opening and closing parentheses \\
D. Specific numeric combinations \\

\item R is case-sensitive

True \\
False \\

\item The expression 2\^{}4 returns:

A. 2 \\
B. 8 \\
C. 16 \\
D. 24 \\

\item To create a vector one would use the following function:

A. vector.new() \\
B. c() \\
C. new.vector() \\
D. C() \\
E. None of the above 

\item To assign the results of a function to an object, you can use which \textbf{two} methods:

A. $->$ \\
B. $=$ \\
C. $==$ \\
D. $<<>>$ \\

\item To determine how many values are in a numeric vector, one would use the following function: 

A. count() \\
B. values() \\
C. length() \\
D. sum() \\

\item To create a data frame we use the following function:

A. data.frame() \\
B. dataframe() \\
C. data.framenew()\\
D. newdataframe() \\

\item To extract the variable ``Temp'' from ``Claremont'' dataframe, you would type

A. Temp\$Claremont \\
B. Temp\$Claremont() \\
C. var(Temp\$Claremont)\\
D. Claremont\$Temp \\

\item We want to select the second column from a data frame called Mangroves, which expression is correct:

A. Mangroves[1,2] \\
B. Mangroves[2] \\
C. Mangroves[,2] \\
D. Mangroves[2,1] 

\item What is the symbol for missing values in R?

A. na \\
B. Na \\
C. NaN \\
D. NA 

\item What will these lines produce?

\begin{quote}
mm $<-$ function(Vmax, Ks, S) \{\\
V $<-$ Vmax* (S/(Ks + S)) \\
return(V)\\
\}

\end{quote}
 
A. I have no idea. \\
B. A function named ``mm''. \\
C. An error message that is nearly impossible to interpret. \\
D. The value of V 

\item To get the path and file name using a popup window, you can use the following function:

A. get.file() \\
B. file.choose() \\
C. find.file() \\
D. file() 

\item To view the first six observations of a dataframe you can use the following function:

A. topsix() \\
B. head() \\
C. show() \\
D. run6() 

\item What functions would you use to get the mean and standard deviation of a vector?

A. average(); stdev() \\
B. average(); sd() \\
C. mean(); stdev() \\
D. mean(); sd() 

\item If you want make a scatter plot you would use the following function:

A. plot.scatter() \\
B. chart() \\
C. barplot() \\
D. plot() 

\item If you wanted to determine if two variables have a linear relationship, you can create a model in R using the following function:

A. model() \\
B. lm() \\
C. correlation() \\
D. linear()

\item What are some characteristics of good graphics?

A. Lack unnecessary ink \\
B. Effective labeling \\
C. Can be easily read at a distance \\
D. Axes are clearly labeled that include units\\
E. The legend is large enough to see and read\\
F. The lines can be interpreted in a color and black and white scheme. \\
G. All of the above. 


\item What is the function to extract the slope and intercept from a linear model?

A. abline()\\
B. coef()\\
C. Coef()\\
D. slopeinter()\\
E. SlopeInter()\\

\item Which function can be used to overlay a line onto a plot?

A. abline()\\
B. bestline()\\
C. fitline()\\
D. drawline()\\
\end{enumerate}

\end{document}
