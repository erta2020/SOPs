% Appendix B 

\subsection{Reading Data into R}

\href{http://thinkertools.org/grainsize/}{Database}

The password is ``18Pomona87''

FileRead function...



Reading .csv Data into R
\begin{Schunk}
\begin{Soutput}
function () 
{
    sample <- read.csv(file.choose())
    reading <- read.csv(file.choose())
    return(reading)
    return(sample)
}
\end{Soutput}
\end{Schunk}


\subsection{Percent Passing Through No. 10 Sieve}

Using the function below, we can calculate the amount of soil that passed through Sieve No. 10. Once the function is call, record the results into Box 16. 

\begin{Schunk}
\begin{Sinput}
> source("/home/CAMPUS/mwl04747/github/SOPs/32_Soil_Texture_Analysis/Soil_functions/PassNo10_fun.R")
> PassNo10
\end{Sinput}
\begin{Soutput}
function (MassPassed, Total) 
{
    round(MassPassed/Total * 100, 1)
}
\end{Soutput}
\end{Schunk}

\begin{Schunk}
\begin{Sinput}
> PassNo10 = function(MassPassed,Total){
+   PercentPassed = MassPassed/Total * 100
+   cat("Enter", round(PercentPassed, 1), "into Box 16")
+ }
\end{Sinput}
\end{Schunk}

The syntax of the function requires two parameters, mass of soil passed (Box 11) and total soil used (Box 10), PassNo10(Mass of Soil Passed, Total Soil Used). 

For example, if we have the mass of the soil that passed through the sieve 122 grams and we had a total to start with of 133 grams.

\begin{Schunk}
\begin{Sinput}
> PassNo10(122, 133)
\end{Sinput}
\begin{Soutput}
Enter 91.7 into Box 16
\end{Soutput}
\end{Schunk}

\subsection{Calculating Soil Mass for Analysis}

The following function can be used to calculate the ...

\begin{Schunk}
\begin{Sinput}
> # Function to Calculate Ws 
> SoilEquiv = function(tin, airdry, ovendry, tinmass, desiredmass = 50){
+ hygro = round(ovendry/airdry, 2)
+ wd = (airdry - ovendry)/(ovendry-tinmass)
+ WS_e = round(desiredmass * (1 + wd), 2)
+ output = data.frame(
+   Parameter = c("Tin ID", "Tin Tare Weight", "Mass of air-dried soil", 
+       "Mass of oven-dried soil", "Hygroscopic Correction Factor", 
+       "Desired Oven-dried Soil", "Effective Soil Weight (WSe)"),
+   Value = c(tin, tinmass, airdry, ovendry, hygro, desiredmass, WS_e),
+   Box = c(12, 13, 14, 15, 17, 9, 18))
+ print(output)
+ }
\end{Sinput}
\end{Schunk}

The function (SoilEquiv) requires four parameters to run provide the output, the mass of soil to be used for the hydrometer test. This mass is the air-dried soil to be used but is equivalent to the desired oven-dried soil, is based on the hygroscopic water methods. 

The function's syntax is as follows: SoilEquiv(Tin \#, Air-Dried Mass, Oven-Dried Mass, Tin Tare, Desired Oven-Dried Soil)

Below is an example, where the $tin number = 4$, $airdry = 27.4$, $ovendry = 25.6$, $tinmass = 12.3$, $desiredmass = 50$ and as used in R: 

\begin{Schunk}
\begin{Sinput}
> SoilEquiv(4, 27.4, 25.6, 12.3, 50)
\end{Sinput}
\begin{Soutput}
                      Parameter Value Box
1                        Tin ID  4.00  12
2               Tin Tare Weight 12.30  13
3        Mass of air-dried soil 27.40  14
4       Mass of oven-dried soil 25.60  15
5 Hygroscopic Correction Factor  0.93  17
6       Desired Oven-dried Soil 50.00   9
7   Effective Soil Weight (WSe) 56.77  18
\end{Soutput}
\end{Schunk}


\subsection{Hydrometer Correction Factor}

Correcting for the meniscus, we use our ``blank'' hydrometer readings to calculate the composite correction factor, CCF. 

\begin{equation}
R_c = R_a - (R_b - 1)
\end{equation}

The equation is implemented in the De function below...



\subsection{Effective Depth}

We have two hydrometer types in the lab -- and the formula to calcutate the effective depth varies based on which hydrometer we use. 

\begin{equation}
\texttt{Effective Depth} = XXX
\end{equation}

\begin{table}
		\begin{tabular}{lcl}\hline
Hydrometer Model	& Code for R function  	&	$R = f(R_a)$	\\ \hline\hline
Heavy Liquid			&		``HL''							&	$16.295 - 0.1645 * R_c$\\
151H Hydrometer		& 	``151H''						& $234.9 + -228.X * R_c$\\ \hline
		\end{tabular}
	\caption{Hydrometer Types and Effective Depth Formula}
	\label{tab:HydrometerTypesAndEffectiveDepthFormula}
\end{table}

Thus, the function syntax is

Hydrometer = ``HL'' or Hydrometer = ``151H''

\subsection{Formatting the Data}

There are two ways to process the data. One, you can analyze the data using a calculator or use excel as a fancy calculator. Either way, we are not using the efficacy of using available software. For our analysis, we will analyze the data using R that is capable of analyzing many samples simultaneously. In order to accomplish this, we need to have the data entered in a consistent format. 

Thus, we have created a spreadsheet or access database? for data entry or shiny apps?

\subsection{Importing data}

importing data into R is not difficult, but there are tricks. The following function, allows the data to be imported into R from a csv file. This function is used to start the analysis. 

\begin{Schunk}
\begin{Sinput}
> import = function(){
+ read.csv(file.choose())
+ }
\end{Sinput}
\end{Schunk}


\subsection{Pre-Processing Data}

Sample processing...

imputing missing values...

\subsection{Calculating the $D_e$ and PF w/K}

Below is the function to calculate the D and PF, where K as a constant is calculated. 

\subsection{Effective Particle Diameter Size, $D_e$}

The formula is 

\begin{equation}
d_e = \sqrt{\frac{30 \eta l}{980 (GS_p - GS_f)* t}}
\end{equation}

Once the data have been inspected for integrity, we can use the following function to calculate the two values of interest: $D_e$ and PF. 

\begin{Schunk}
\begin{Sinput}
> SampleParams = data.frame(Ws_actual = 50, Hygro = .95, GSp = 2.65); SampleParams
> 
> 
\end{Sinput}
\end{Schunk}

\begin{Schunk}
\begin{Sinput}
> GSp = 2.65
> GSf = 1
> t = c(0.5, 1, 3, 10, 30, 60, 90, 120, 1440)
> #c(2, 5, 15, 30, 60, 250, 1440)
> L1 = c(1.01, 1.01, 1.008, 1.007, 1.007, 1.0065, 1.0065, 1.0064, 1.0055)
> # Heavy Density Hydrometer
> Vb = 40
> A = 27.8
> L2 = 14.5
> effective_depth = L1 + 0.5*(L2 - Vb/A) # result in centimeters
> C1 = 0.01732483379693
> C2 = -5.04157465609E-4
> C3 = 8.387438669317E-6
> C4 = -7.401129271698E-8
> C5 = 2.625994080072E-12
> eta = C1 + T * (C2 + T * (C3 + T *  (C4 + T * C5)))
> d_e = sqrt((30 * eta * effective_depth)/((980*GSp - GSf) *t)); d_e
> 
> #{\frac{30 \eta l}{980 (GS_p - GS_f)* t}}
> 
\end{Sinput}
\end{Schunk}

\begin{equation}
PF = [(100,000/M_s)[m|5]P5 \frac{GS_p}{GS_p - GS_w}](R_c - GS_w)
\end{equation}

\begin{Schunk}
\begin{Sinput}
> M_s = 60
> PF = 100000/M_s * (GSp)/(GSp-GSf)*(L1 - GSf); PF
> plot(d_e, PF)
\end{Sinput}
\end{Schunk}

\subsection{Determine the Texture Classification}

\subsection{Program to Process Multiple Samples}

To process the analysis, we need to create a standardized electronic data entry template that can later be read by R, which will conduct the analysis above. R can be used to automate the analysis and process multiple samples at the same time. Below is the program that I have developed for this purpose. 

First, use the data entry template...

 

where $M_s$ = a biased soil sample value, that includes the amount of hydroscopic water and the percentage of soil that passed through a 2 mm sieve. 


\section{Calculations with $\eta$ and $GS_w$}

\subsection{Calculating fluid viscosity, $\eta$}

The fluid viscosity, $\eta$, various with temperature. We will use a deterministic equation to estimate the viscosity:

\begin{equation}
\eta = C_1 + T * (C_2 + T * (C_3 + T *  (C_4 + T * C_5)))
\end{equation}

where T is the fluid temperature ($^\circ$ C),
$C_1$ = 0.01732483379693,
$C_2$ = -5.04157465609E-4
$C_3$ = 8.387438669317E-6
$C_4$ = -7.401129271698E-8
$C_5$ = 2.625994080072E-12.

This equation is automatically implemented in the De function as described below. 

\subsection{Calculating Fluid Density, $\rho_f$)}

Similar to fluid viscosity, fluid density, $\rho_f$ also changes with temperature,

\begin{equation}
G_w = C1 + T * (C2 + T * (C3 + T * C4))
\end{equation}

where T is temperature, 
$C_1$ = 0.99991003252,
$C_2$ = 0.00005201921, 
$C_3$ = -0.00000751229, 
$C_4$ = -0.00000003605183,




Some random example...


\subsection{Effective Particle Diameter Size, $D_e$}

\begin{equation}
P_s = \sqrt{\frac{30*\eta*l}{980*(G_s - G_w)*E_t}}
\end{equation}

\newpage
\subsection{$D_e$ using K}

\begin{equation}
D_e = K \sqrt{L/T}
\end{equation}


Calculating K

\begin{Schunk}
\begin{Sinput}
> 
\end{Sinput}
\end{Schunk}

One can easily use the function with the temperature and have K returned: 

\begin{Schunk}
\begin{Sinput}
> 
\end{Sinput}
\end{Schunk}

Calculating L

\begin{Schunk}
\begin{Sinput}
> 
\end{Sinput}
\end{Schunk}

Calculating $D_e$

\begin{equation}
D_e = K * sqrt{L/T}
\end{equation}

\begin{Schunk}
\begin{Sinput}
> 
\end{Sinput}
\end{Schunk}

\subsection{Alternative Calculation w/o K}

\begin{Schunk}
\begin{Sinput}
> # Fluid Viscosity
> C1 = 0.01732483379693
> C2 = -5.04157465609E-4
> C3 = 8.387438669317E-6
> C4 = -7.401129271698E-8
> C5 = 2.625994080072E-12
> eta <- function(T){
+ C1 + T * (C2 + T * (C3 + T * (C4 + T * C5)))
+ } 
\end{Sinput}
\end{Schunk}

Now, we calculate the fluid viscosity, $\eta$, 

\begin{Schunk}
\begin{Sinput}
> eta(23)
\end{Sinput}
\begin{Soutput}
[1] 0.009266407
\end{Soutput}
\end{Schunk}


And the specific gravity of water, $Gs_w$, 

\begin{Schunk}
\begin{Sinput}
> Gs_w <- function(T){
+ C1 + T * (C2 + T * (C3 + T * C4))}
\end{Sinput}
\end{Schunk}

%id = 18

%plot(de[id,], Pf[id,], ty='b', las=1, log='x', xaxt='n', xlim=c(1, 0.00001), ylab = "Percent Finer", xlab="Particle Size (mm)", main='Particle Size Distribution', panel.first=grid())
%ticks <- seq(-5, 0, by=1)
%labels <- sapply(ticks, function(i) as.expression(bquote(10^ .(i))))
%axis(1, at=c(0.00001, 0.0001, .001, 0.01, 0.1, 1.0), labels=labels)

%# Minor Grids
%# n <- n+2
%#  minors <- log10(pretty(10^major.ticks[1:2],n))-major.ticks[1]
%# minors <- minors[-c(1,n)]
%# minor.ticks = c(outer(minors,major.ticks,`+`))
%# minor.ticks <- minor.ticks[minor.ticks > mn & minor.ticks < mx]
%# axis(ax,at=minor.ticks,tcl=par("tcl")*t.ratio,labels=FALSE)

%abline(v=0.002, col='Red'); text(x = .0001, y = Pf[id,7], 'Clay', col='red', cex=1.25, pos=3)
%abline(v=0.05, col='Blue'); text(x = .01, y = Pf[id,4], 'Silt', col='blue', cex=1.25, pos=1)
%abline(v=2.0, col='Green'); text(x = 0.50, y = Pf[id,1], 'Sand', col='green', cex=1.25, pos=1)


%#sand = approx(x, y = NULL, xout, method = "linear", n = 50, yleft, yright, rule = 1, f = 0, ties = mean)
